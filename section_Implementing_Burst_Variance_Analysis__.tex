\section{Implementing Burst Variance Analysis (BVA) with FRETBursts}

FRET histograms could show more information than just mean FRET efficiencies. Broad FRET distributions might be attributed to 1) the mixture of multiple species with static but different FRET efficiencies, 2) single species with dynamic fluctuations between multiple FRET states, or 3) a combination of the two cases. Burst Variance Analysis (BVA) is an analysis method for single molecule FRET experiments, developed to detect molecular dynamics~\cite{Torella_2011}. It has been successfully implemented to identify heterogeneities in FRET histograms due to dynamic processes of biomolecules in millisecond time scale~\cite{Torella_2011, Robb_2013}.

The concept of BVA is comparing an expected standard deviation ($\sigma_E$) for a burst to calculated standard deviations ($s_E$)of sub-bursts comprised of \textit{n} consecutive photons within the burst in order to figure out whether the observed broadness of FRET distributions are owing to the molecular dynamics~\cite{Torella_2011}. If the broadness originates from different molecules with static but different FRET efficiencies, the expected standard deviations of each individual molecule only depends on shot noise limited photon statics and the ones for sub-bursts will follow a binomial distribution.  

$$ n_t = n_a + \gamma\,n_d$$ 

see section~\ref{sec:ph_streams}