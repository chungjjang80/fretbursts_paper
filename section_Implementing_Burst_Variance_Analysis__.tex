\section{Implementing Burst Variance Analysis (BVA) with FRETBursts}

FRET histograms could show more information than just mean FRET efficiencies. Broad FRET distributions might be attributed to 1) the mixture of multiple species with static but different FRET efficiencies, 2) single species with dynamic fluctuations between multiple FRET states, or 3) a combination of the two cases. Burst Variance Analysis (BVA) is an analysis method for single molecule FRET experiments, developed to detect molecular dynamics~\cite{Torella_2011}. It has been successfully implemented to identify heterogeneities in FRET histograms due to dynamic processes of biomolecules in millisecond time scale~\cite{Torella_2011, Robb_2013}.

The concept of BVA is to figure out whether the observed broadness of FRET distributions are owing to the molecular dynamics, by comparison of a calculated standard deviation ($s_E$) of FRET efficiencies for sub-bursts ,comprised of \textit{n} consecutive photons within the burst, from a mean FRET efficiency for a burst to an expected standard deviation based on shot noise limited distribution~\cite{Torella_2011}. 

If the observed broadness originates from different molecules having distinct FRET efficiencies without dynamics, $s_E$ of each burst is only affected by shot noise and will follow the expected standard deviation curve rationalized by a binomial distribution (see equation 4 in~\cite{Torella_2011}). However, if the observed broadness is due to dynamics of single species of biomolecules, $s_E$ of each burst has to be larger than the expected standard deviation and sits on above the expected standard deviation curve (see figure ).  



see section~\ref{sec:ph_streams}