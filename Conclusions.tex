\section{Conclusions}
\label{sec:conclusions}

FRETBursts provides an open source implementation of state-of-the-art smFRET burst analysis 
accessible to the whole single-molecule community.
FRETBursts implements several novel concepts which can lead 
to significantly more accurate results in specific situations:
time-dependent background estimation, background dependent burst search threshold,
burst weighting, burst selection based on γ-corrected burst sizes.

More importantly, FRETBursts provides a library of well-tested routines
for timestamps and burst manipulation, making it an ideal environment to 
quickly develop and compare novel analytical techniques.

We summarize here what we consider to be the strengths
of the FRETBursts software.

\begin{enumerate}
\item Open source and openly developed. The source code can be checked, modified and
adapted for different purposes. All the software dependencies are open source as well.
\item Several state-of-the-art and novel algorithms for each step of the
smFRET burst analysis pipeline are provided.
\item Modern software engineering design: the
\href{http://en.wikipedia.org/wiki/Don\%27t_repeat_yourself}{DRY principle}
to reduce duplication and the
\href{http://en.wikipedia.org/wiki/KISS_principle}{KISS principle}
to avoid over-engineering were followed.
\item Defensive programming~\cite{Prli__2012}: care about code readability,
unit and regression testing and continuous integration.
\end{enumerate}

Given its features, FRETBursts is both a toolkit for developing novel algorithms
in smFRET burst analysis and a well-tested software for standard smFRET burst analysis. 
Its open source nature and open development process assures that discovered bug 
are always timely disclosed, and lowers the barriers
for new users to identify and potentially fix bugs, and even develop new or improved analysis.

\section*{Acknowledgments}
This work was supported in part by National Institutes of Health (NIH)
grant R01-GM95904 and by U.S. Department Energy (DOE) grant DEFC02-02ER63421-00.
