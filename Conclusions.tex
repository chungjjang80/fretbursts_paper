\section{Conclusions}
\label{sec:conclusions}

In this paper we have examined a standard smFRET burst analysis, describing
the different steps as performed by FRETBursts.
We have highlighted several novel concepts implemented in FRETBursts which
can lead to significantly more accurate results in specific situations:
time-dependent background estimation, background dependent burst search threshold,
burst weighting, burst selection based on gamma corrected burst sizes.

We conclude enumerating what we consider to be the strengths
of this software.

\begin{enumerate}
\item Open source and openly developed. The source code can be checked, modified and
adapted for different purposes. All the software dependencies are open source as well.
\item Several state-of-the-art and novel algorithms for each step of the
smFRET burst analysis pipeline are provided.
\item Modern software engineering design: the
\href{http://en.wikipedia.org/wiki/Don\%27t_repeat_yourself}{DRY principle}
to reduce duplication and the
\href{http://en.wikipedia.org/wiki/KISS_principle}{KISS principle}
to avoid over-engineering were followed.
\item Defensive programming~\cite{Prli__2012}: maximum code readability,
extensive unit and regression testing and continuous integration.
\end{enumerate}

Given these features FRETBursts is suitable both as a toolkit to develop novel algorithms
in smFRET burst analysis and as a software for processing smFRET data files with
state-of-the art algorithms.

