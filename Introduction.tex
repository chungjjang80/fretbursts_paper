\section{Introduction}

\subsection{smFRET and burst analysis}

FRETBursts is an open source python software for burst analysis of confocal 
single-molecule FRET (smFRET) data. 
It is hosted and and openly developed~\cite{Prli__2012} on GitHub, where
issues, feature requests and contributions can be reported.

In this paper we provide an broad overview of FRETBursts. The 
paper is structured as follows. In section~\ref{sec:concepts} we
introduce a few preliminary concepts and some specific terminology.
In the next subsections mention how to install~(\ref{sec:install}) and 
load~(\ref{sec:import}) FRETBursts.
In section~\ref{sec:analysis} we illustrate how to execute the several steps involved
in smFRET burst analysis: data loading (section~\ref{sec:dataload}), defining the 
alternation periods (section~\ref{sec:alternation}), background 
correction (section~\ref{sec:bg_calc}), burst search (section~\ref{sec:burstsearch}), 
burst filtering (section~\ref{sec:burstsel}), FRET fitting (section~\ref{sec:fretfit}). 
For each step different approached and options are discussed.

In addition to this paper, we refer the interested readers to the 
\href{http://nbviewer.ipython.org/github/tritemio/FRETBursts_notebooks/blob/master/notebooks/FRETBursts\%20-\%20us-ALEX\%20smFRET\%20burst\%20analysis.ipynb}{FRETBursts µs-ALEX tutorial} 
for a single complete example of µs-ALEX data analysis and to the
\href{http://fretbursts.readthedocs.org/}{FRETBursts Reference Documentation}
for a detailed description of each function.

\subsection{Installing FRETBursts}
\label{sec:install}
FRETBursts is a standard python package that requires the 
\href{http://www.scipy.org/stackspec.html}{"scipy stack"}, a 
standard set of scientific python libraries.
The scipy stack is easily installed by installing a free scientific python
distribution such as \href{https://store.continuum.io/cshop/anaconda/}{Continuum Anaconda}, 
although some users may prefer another distribution or a manual installation.

FRETBursts itself can be installed through the standard python package manager (PIP)
with the command \texttt{pip install fretbursts}. 
Alternatively the latest development version can be downloaded directly 
from GitHub. For more information on different installation methods refer
to the 
\href{http://fretbursts.readthedocs.org/en/latest/installation.html}{FRETBursts reference documentation}.

\subsection{Executing FRETBursts}
\label{sec:import}
In general, we suggest to import FRETbursts with the expression:

\begin{verbatim}
>>> import fretbursts as fb
\end{verbatim}

after which all the FRETBursts functions will be availabe under the \verb|fb.|
prefix. In this article, for brevity, we assume that FRETBursts is imported 
with the shortcut form instead:

\begin{verbatim}
>>> from fretbursts import *
\end{verbatim}

that allows skiping the \verb|fb.| prefix and it will also import some common 
numeric libraries (i.e. \textit{numpy} and \textit{matplotlib.pyplot} 
imported as \verb|np| and \verb|plt| respectively).

In order to enable the reproducibility of the analysis and to ease
the sharing of the complete workflow we encourage using FRETBursts through 
the IPython Notebook environment. 
IPython notebooks are documents in which the code is interspersed with 
prose in a web-based format that can contain math formulas, hyperlinks 
and figures (or anything else a web browser can display).
Furthermore, the "notebook workflow"\cite{Shen_2014} has the advantage 
of automatically recording not only the all analysis commands and parameters 
but also the data file names, software versions and the full output 
(figures, tables, etc...) in a single document that can be interactively 
modified and re-executed.

All the FRETBursts tutorials are 
ipython notebooks and, in fact, the preferred way to start a new analysis is 
copying and modifying a pre-existing FRETBursts notebook.
