\section{Introduction}

\subsection{What is FRETBursts}

In the last 20 years, single molecule FRET (smFRET), has emerged as one of the most
useful techniques in single-molecule spectroscopy~\cite{Weiss_1999},\cite{Hohlbein_2014}. 
As opposed to ensemble experiments, smFRET allows to resolve conformational 
changes of biomolecules or measure binding-unbinding kinetics on heterogeneous samples. 
smFRET measurements on freely diffusing molecules (the focus of this paper) have the advantage 
of probing molecules and processes without perturbation from surface immobilization~\cite{Dahan_1999}\cite{Eggeling_1998}. 

The field of freely-diffusing smFRET data analysis, has seen a number of significant 
contributions over the years~\cite{Fries_1998}\cite{Eggeling_2001}\cite{Zhang_2005}\cite{Gopich_2005}\cite{Lee_2005}\cite{Nir_2006}\cite{Antonik2006}\cite{Gopich_2007}\cite{Gopich_2008}\cite{Camley_2009}\cite{Santoso_2010}\cite{Torella_2011}\cite{Tomov_2012}. However, so far, except for the fundamental steps of burst analysis, 
there is no single approach that is universally accepted and broadly applied. 
On one hand this situation stems from the fact that
different approaches tend to answer slightly different questions.
On the other hand, this is the result of the trade off in each research lab
between accuracy and software complexity, in particular on the amount of effort
each group wants to invest in implementing new non-trivial methods reported in literature.

Currently, in fact,
each research group have reimplemented its own unpublished or closed-source versions
of the analysis software, with very little collaboration or code sharing.
Even within our group, smFRET papers merely mention that custom-made software has been
used with very little additional details (\cite{Lee_2005}\cite{Nir_2006}).
The fact that setups for freely-diffusing smFRET can significantly 
vary (in number of polarization or spectral channels for example), 
makes the problem only worst.
This situation, represents a real impediment to the scientific progress because:

(a) as new methods are proposed in literature, understandably, only few 
groups are willing to invest the time necessary to reimplement it within 
the internally used software. In the rare occasions when the new methods 
is really groundbreaking (most of the science is incremental), the reimplementation requires 
a huge duplication of efforts, and the correctness of the implementation
is difficult to validate.

(b) the differences in implementation details among the various software
are enough that a direct comparison of their results is very difficult. 
This limits the ability to cross-validate the correctness of different 
implementations of the same method or to compare the accuracy 
of different methods.

From a more general stance, since the pioneering work of Donoho in the 90'~\cite{Buckheit_1995}, 
it is widely understood that developing and maintaining open source scientific software
for reproducible research it is a critical requirement for progress in modern 
science. Peer-reviewed publications describing such software are therefore necessary~\cite{Pradal_2013}, 
although there still a open debate on an effective model for peer-reviewing this
class of publications~\cite{Check_Hayden_2015}.

Facing the previous issues, we decided to develop FRETBursts, 
an open source python software for burst analysis of freely-diffusing
single-molecule smFRET experiments. 
With FRETRBursts we provide a tool that is available to any scientist
to use, study and modify. Furthermore, the execution model has been though
to facilitate computational reproducibility.
FRETBursts is hosted and openly developed on GitHub~\cite{Prli__2012}, 
where users can send comments, report issues or contribute code.

Understanding smFRET burst analysis requires several concepts and definitions.
In this paper we aim to provide a brief introduction to smFRET analysis concepts
and terminology used by FRETBursts. We illustrate how to perform
the fundamental steps of burst analysis, highlighting key parameters
and algorithms available. We will not cover extensively all the FRETBursts
features and options. For more detailed information, the interested readers can refer
to the FRETBursts Reference Documentation or can ask question about FRETBursts use by opening 
a new issue on GitHub.

\subsection{Paper overview}

The paper is structured as follows.
In the next section~\ref{sec:overview} we give an overview of the software features,
the modality of execution and the development style.
In section~\ref{sec:concepts}, we
introduce a few preliminary concepts and some specific terminology needed
to understand the smFRET burst analysis.
In section~\ref{sec:analysis}, we detail the execution the several steps involved
in smFRET burst analysis: data loading (section~\ref{sec:dataload}), defining
excitation alternation periods (section~\ref{sec:alternation}), background
correction (section~\ref{sec:bg_calc}), burst search (section~\ref{sec:burstsearch}),
burst selection (section~\ref{sec:burstsel}) and FRET fitting (section~\ref{sec:fretfit}).
The aim is to provide enough information to understand the specificities of
the different algorithms and to be able to adapt the analysis to new situations.
Finally, in section~\ref{sec:conclusions}, we summarize what we believe to be
the strengths of FRETBursts software.

In addition to this paper, we refer the interested readers to the
\href{http://nbviewer.ipython.org/github/tritemio/FRETBursts_notebooks/blob/master/notebooks/FRETBursts\%20-\%20us-ALEX\%20smFRET\%20burst\%20analysis.ipynb}{FRETBursts µs-ALEX tutorial}
for a single complete example of µs-ALEX data analysis and to the
\href{http://fretbursts.readthedocs.org/}{FRETBursts Reference Documentation}
for a more in-depth description of the software. The reference documentation
includes also a special section describing how to install and run the software
for users without any prior python knowledge.

Finally, in order to make this paper accessible to the widest number of readers,
we concentrated python-specific details in special subsections titled
\textit{Python details}. These subsections provide deeper insights for readers
already familiar with python and can be safely skipped otherwise. Note also
that all commands shown in this paper can be found in the FRETBursts tutorial notebook
and, in general, do not need to be re-typed.

