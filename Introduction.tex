\section{Introduction}

FRETBursts is a python package for burst analysis of confocal single-molecule FRET (smFRET) data.

\subsection{Installation}
FRETBursts is a standard python package that requires the "scipy stack", a set of core scientific python packages.
The "scipy stack" is easily installed through a free scientific python distribution such as Continuum Anaconda, although some users may prefer another distribution or a manual installation.

FRETBursts can be installed through the standard python package manager (PIP) with 
the command \verb|pip install fretbursts|. Alternatively the latest development version can be installed from GitHub.
For more information on different installation methods see the \href{http://fretbursts.readthedocs.org/en/latest/installation.html}{FRETBursts documentation}.

\subsection{Executing FRETBursts}
For interactive work, we strongly encourage to use FRETBursts through the IPython Notebook environment. All the FRETBursts tutorials are ipython notebook documents and, indeed, a quick way to start a new analysis is copying a pre-existing FRETBursts notebook and modifying it.

The "notebook workflow"\cite{Shen_2014} has the advantage to automatically record all the analysis steps including
data file names, software versions, analysis details and the full output (figures, tables, etc...).
