\section{Introduction}

\subsection{smFRET and burst analysis}

FRETBursts is an python software for burst analysis of confocal single-molecule
FRET (smFRET) data. It is hosted and and openly developend on GitHub, where
issues, feature requests and contributions can be reported.

In this paper we provide an broad overview of FRETBursts. The 
paper is structured as follows. In section~\ref{sec:concepts} we
introduce a few preliminary concepts and some specific terminology.
In section~\ref{sec:analysis} we illustrate how to execute the several steps involved
in smFRET burst analysis: data loading (section~\ref{sec:dataload}), defining the 
alternation periods (section~\ref{sec:alternation}), background 
correction (section~\ref{sec:bg_calc}), burst search (section~\ref{sec:burstsearch}), 
burst filtering (section~\ref{sec:burstsel}), FRET fitting (section~\ref{sec:fretfit}). 
For each step different approached and options are discussed.

In addition to this paper, we refer the interested readers to the 
\href{http://nbviewer.ipython.org/github/tritemio/FRETBursts_notebooks/blob/master/notebooks/FRETBursts\%20-\%20us-ALEX\%20smFRET\%20burst\%20analysis.ipynb}{FRETBursts µs-ALEX tutorial} 
for a single complete example of µs-ALEX data analysis and to the
\href{http://fretbursts.readthedocs.org/}{FRETBursts Reference Documentation}
for a detailed description of each function.

\subsection{Installing FRETBursts}
FRETBursts is a standard python package that requires the "scipy stack", a set
of core scientific python libraries.
The "scipy stack" is easily installed by installing a free scientific python
distribution such as Continuum Anaconda, although some users may prefer another
distribution or a manual installation.

FRETBursts itself can be installed through the standard python package manager (PIP)
with the command \texttt{pip install fretbursts}. 
Alternatively the latest development version can be downloaded directly 
from GitHub. For more information on different installation methods see the
\href{http://fretbursts.readthedocs.org/en/latest/installation.html}{FRETBursts
documentation}.

\subsection{Executing FRETBursts}
In general, we suggest to import FRETbursts with the expression:

\begin{verbatim}
>>> import fretbursts as fb
\end{verbatim}

that will make available all the FRETBursts functions with a concise \verb|fb.|
prefix. In this article, for brevity, we assume that FRETBursts is imported with the
shortcut form:

\begin{verbatim}
>>> from fretbursts import *
\end{verbatim}

that allows to skip the \verb|fb.| prefix and it also imports some common numeric
libraries (\textit{numpy} and \textit{matplotlib.pyplot} imported as 
\verb|np| and \verb|plt| respectively).

We encourage using FRETBursts through the IPython Notebook environment. 
IPython notebooks are web browser documents that contain
both code cells, rich text and multimedia content. In a notebook, the code is 
interspersed with prose in a structured document that can contain math 
formulas, hyperlinks and figures (or anything else a web browser can display).
Furthermore, the "notebook workflow"\cite{Shen_2014} has the advantage 
of automatically recording not only the all analysis commands and parameters 
but also the data file names, software versions and the full output 
(figures, tables, etc...) in a single document that can be interactively 
modified and re-executed.

All the FRETBursts tutorials are 
ipython notebooks and, in fact, the suggested way to start a new analysis is 
copying and modifying a pre-existing FRETBursts notebook.
