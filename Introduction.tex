\section{Introduction}

\subsection{smFRET and burst analysis}

FRETBursts is an open source python software for burst analysis of confocal 
single-molecule Förster Resonance Energy Transfer (smFRET) data. The software can analyze smFRET measurements
with one or several excitation spots~\cite{Ingargiola_2013}. The supported 
excitation schemes include single lasers, Alternating Laser EXcitation (ALEX) with either two CW lasers (µs-ALEX \cite{Kapanidis_2005}) 
or two interleaved pulsed lasers (ns-ALEX \cite{16287971} or Pulsed-Interleaved Excitation (PIE) \cite{M_ller_2005}). 
FRETBursts is hosted and and openly developed~\cite{Prli__2012} on GitHub, where
issues, feature requests and contributions can be reported.

In this paper we provide a broad overview of FRETBursts. 
The paper is structured as follows. 
In the next subsections mention how to install~(\ref{sec:install}) and 
load~(\ref{sec:import}) FRETBursts.
In section~\ref{sec:concepts} we
introduce a few preliminary concepts and some specific terminology.
In section~\ref{sec:analysis} we illustrate how to execute the several steps involved
in smFRET burst analysis: data loading (section~\ref{sec:dataload}), defining 
excitation alternation periods (section~\ref{sec:alternation}), background 
correction (section~\ref{sec:bg_calc}), burst search (section~\ref{sec:burstsearch}), 
burst filtering (section~\ref{sec:burstsel}) and FRET fitting (section~\ref{sec:fretfit}). 
For each step different approaches and options are discussed. 
In section~\ref{sec:conclusions} we conclude by stating what we believe are
the strengths of this software.

In addition to this paper, we refer the interested readers to the 
\href{http://nbviewer.ipython.org/github/tritemio/FRETBursts_notebooks/blob/master/notebooks/FRETBursts\%20-\%20us-ALEX\%20smFRET\%20burst\%20analysis.ipynb}{FRETBursts µs-ALEX tutorial} 
for a single complete example of µs-ALEX data analysis and to the
\href{http://fretbursts.readthedocs.org/}{FRETBursts Reference Documentation}
for a more in-depth description of the software.

\subsection{Installing FRETBursts}
\label{sec:install}
Installing FRETBursts is analogous to installing any standard python
package. A python environment needs to be installed, preferentially
through a scientific python distribution which conveniently include 
all the major scientific libraries.

For new python users we suggest installing 
\href{https://store.continuum.io/cshop/anaconda/}{Continuum Anaconda}
python distribution and then install (or update) FRETBursts with the command:

\begin{verbatim}
conda install fretbursts -c tritemio
\end{verbatim}

For more information on different installation methods refer to the 
\href{http://fretbursts.readthedocs.org/en/latest/getting_started.html}{Getting Started}
section in the FRETBursts documentation.

FRETBursts depends on a standard set of scientific 
python libraries, namely, numpy/scipy for numerical computations, 
matplotlib and seaborn for plotting. Optionally, a few core functions are
optimized with cython and the software is preferentially run through
a Jupyther (IPython) Notebook (optional).
For a complete list of dependencies can be found in the 
\href{http://fretbursts.readthedocs.org/en/latest/getting_started.html}{Getting Started}
section in the FRETBursts documentation.

\subsection{Running FRETBursts}
The preferential way to execute FRETBurts is by running one of the
FRETBursts tutorials which are in the form of Jypyther notebooks.
Jupyther (formerly IPython) notebooks are documents, 
accessed through a web browser, that contain both code and 
rich text (including formulas, hyperlinks, figures, etc...).
FRETBursts tutorials are notebook that can be re-executed,
modified or used to process new data files with minimal modifications.
The "notebook workflow"\cite{Shen_2014} not only eases 
the analysis comprehension (by integrating the cone in a rich document)
but also greatly enhance its reproducibility by saving an execution trail
including software versions, input files, parameters, commands and all
the analysis results (text, figures, tables, etc...).

Given the modality of execution, running FRETBurst does not require
any prior python knowledge. The user only needs familiarize with the
notebook graphical environment to be able to navigate and run the notebooks.
