\section{Introduction}

\subsection{smFRET and burst analysis}

FRETBursts is a python package for burst analysis of confocal single-molecule FRET 
(smFRET) data.

\textit{Expand abstract to introduce smFRET and what is burst analisys}.

\subsection{Installing FRETBursts}
FRETBursts is a standard python package that requires the "scipy stack", a set of core 
scientific python packages.
The "scipy stack" is easily installed through a free scientific python distribution such as Continuum Anaconda, although some users may prefer another distribution or a manual installation.

FRETBursts can be installed through the standard python package manager (PIP) with 
the command \texttt{pip install fretbursts}. Alternatively the latest development version can be installed from GitHub.
For more information on different installation methods see the \href{http://fretbursts.readthedocs.org/en/latest/installation.html}{FRETBursts documentation}.

\subsection{Executing FRETBursts}
In general, we suggest to import FRETbursts with the expression:

\begin{verbatim}
>>> import fretbursts as fb
\end{verbatim}

that will make available all the FRETBursts functions with a concise `fb.` prefix. In this article, however, we assume that FRETBursts is imported with the shortcut form:

\begin{verbatim}
>>> from fretbursts import *
\end{verbatim}

that allows to skip the \verb|fb.| prefix and also imports some common numeric libraries (numpy and matplotlib.pyplot imported as \verb|np| and \verb|plt| respectively).

Furthermore we encourage to use FRETBursts through the IPython Notebook environment. All the FRETBursts tutorials are ipython notebook documents and, indeed, a quick way to start a new analysis is copying a pre-existing FRETBursts notebook and modifying it.

The "notebook workflow"\cite{Shen_2014} has the advantage of automatically recording all the analysis steps including
data file names, software versions, analysis details and the full output (figures, tables, etc...).


