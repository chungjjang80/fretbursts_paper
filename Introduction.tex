\section{Introduction}

\subsection{What is FRETBursts}

FRETBursts is an open source python software for burst analysis of freely-diffusing 
single-molecule Förster Resonance Energy Transfer (smFRET) data. 
The software can analyze smFRET measurements
with one or multiple excitation spots~\cite{Ingargiola_2013}. The supported 
excitation schemes include single laser, Alternating Laser Excitation (ALEX) 
with either CW lasers (µs-ALEX \cite{Kapanidis_2005}) 
or interleaved pulsed lasers (ns-ALEX~\cite{Laurence_2005} or 
Pulsed-Interleaved Excitation (PIE) \cite{M_ller_2005}). 
FRETBursts is hosted and and openly developed~\cite{Prli__2012} on GitHub, where
issues, feature requests and contributions can be reported.

\subsection{Features overview}

FRETBursts implements both classical and novel algorithms for smFRET data analysis 
including background estimation as a function of time (including background accuracy 
metrics), sliding-window burst search~\cite{Eggeling_1998}, dual-channel bursts search~\cite{Nir_2006} and
generic and composable burst selection methods based on arbitrary criteria 
(including a large set of pre-defined selection rules). Novel features include burst size
selection with gamma-corrected burst size, burst weighting, burst search with background-
dependent threshold (in order to guarantee a minimal single-to background ratio~\cite{Michalet_2012}).
Finally a large novel set of FRET distribution fitting options are provided including
FRET distribution fitting through (1) histogram fitting (with arbitrary model functions), 
(2)  non-parametric weighted \textit{Kernel Density Estimation} (KDE), (3) weighted 
Expectation Maximization (EM), (4) Maximum Likelihood fitting using Gaussian models 
or Poisson statistic. FRETBursts includes also a large number of
predefined and customizable plot functions that (thanks to the \textit{matplotlib} 
graphic library) produce publication quality plots in a wide range of formats.

\subsection{Development style and open source}

FRETBursts (and the entire python ecosystem it depends upon) is open source 
and the source code is fully available to any scientist for studying, 
inspection and modifications.
The authors welcome users to use GitHub Issues for questions, discussions
and bug reports, and to submit patches through GitHub Pull Requests.

In order to minimize the chance of bugs and erroneous results, FRETBursts is developed
following modern software engineering techniques such 
as defensive programming~\cite{Wilson_2014}, unit testing, regression testing and continuous integration.

The open source nature of FRETBursts and of the python ecosystem 
not only make it a more transparent, reviewable, platform 
for scientific data analysis, but also allows 
to leverage state-of-the-art online services like GitHub for hosting, issues tracking and code 
reviews, TravisCI for continuous integration (i.e. automated the test suite execution after 
each commit on multiple platforms) 
and ReadTheDocs.org for automatic documentation building and hosting. 
All these services would be extremely costly, if at all available, 
for a proprietary software or platform.

\subsection{FRETBursts and Jupyter Notebook}

The preferential way to execute FRETBursts is executing one of the tutorials 
which are in the form of \href{http://ipython.org/notebook.html}{Jypyter notebooks}.
Jupyter (formerly IPython) notebooks are web-based documents that contain both 
code and rich text (including equations, hyperlinks, figures, etc...).
FRETBursts tutorials are notebook that can be re-executed,
modified or used to process new data files with minimal modifications.
The "notebook workflow"\cite{Shen_2014} not only facilitates 
the description of the analysis (by integrating the code in a rich document)
but also greatly enhance its reproducibility by storing an execution trail
that includes software versions, input files, parameters, commands and all
the analysis results (text, figures, tables, etc...).

Given the modality of execution, running FRETBurst does not require
any prior python knowledge. The user only needs familiarize with the
notebook graphical environment to be able to navigate and run the notebooks.
The list of FRETBursts notebooks can be found in the 
\verb|FRETBursts_notebooks| repository on GitHub.

\subsection{Paper overview and further resources}

Although running the software is technically simple, understanding the smFRET 
burst analysis requires several concepts and definitions.
In this paper we aim to provide a brief introduction to smFRET analysis concepts
and terminology used by FRETBursts. Furthermore we illustrate how to perform the
the fundamental steps of burst analysis, highlighting the important parameters
and algorithms that can be chosen. We will not cover extensively all the FRETBursts
features and options. For more detailed information, the interested readers can refer 
to the FRETBursts Reference Documentation and to the source code (that we strive 
to make readable and well commented). 
Furthermore, users can ask question about FRETBursts usage opening a new issue on GitHub.

The present paper is structured as follows. 
In the next subsections we briefly mention how to install~(\ref{sec:install}) and 
load~(\ref{sec:import}) FRETBursts.
In section~\ref{sec:concepts} we
introduce a few preliminary concepts and some specific terminology needed 
to understand the smFRET burst analysis. 
In section~\ref{sec:analysis} we detail the execution the several steps involved
in smFRET burst analysis: data loading (section~\ref{sec:dataload}), defining 
excitation alternation periods (section~\ref{sec:alternation}), background 
correction (section~\ref{sec:bg_calc}), burst search (section~\ref{sec:burstsearch}), 
burst filtering (section~\ref{sec:burstsel}) and FRET fitting (section~\ref{sec:fretfit}).
The aim is providing enough information to understand the specificities of 
the different algorithms and to be able to adapt the analysis to new situations.
Finally, in section~\ref{sec:conclusions}, we summarize what we believe to be
the strengths of FRETBursts software.

In addition to this paper, we refer the interested readers to the 
\href{http://nbviewer.ipython.org/github/tritemio/FRETBursts_notebooks/blob/master/notebooks/FRETBursts\%20-\%20us-ALEX\%20smFRET\%20burst\%20analysis.ipynb}{FRETBursts µs-ALEX tutorial} 
for a single complete example of µs-ALEX data analysis and to the
\href{http://fretbursts.readthedocs.org/}{FRETBursts Reference Documentation}
for a more in-depth description of the software.

\subsection{Installing FRETBursts}
\label{sec:install}
Installing FRETBursts is analogous to installing any standard python
package. The python interpreter needs to be installed, preferentially
through a scientific python distribution which conveniently include 
all the major scientific libraries.

For new python users we suggest installing 
\href{https://store.continuum.io/cshop/anaconda/}{Continuum Anaconda}
python distribution and then install (or update) FRETBursts with the command:

\begin{verbatim}
conda install fretbursts -c tritemio
\end{verbatim}

For more information on different installation methods refer to the 
\href{http://fretbursts.readthedocs.org/en/latest/getting_started.html}{Getting Started}
section in the FRETBursts documentation.

FRETBursts depends on a standard set of scientific 
python libraries, namely, \href{http://www.numpy.org/}{numpy}/\href{http://www.scipy.org/}{scipy} for numerical computations, 
\href{http://matplotlib.org/}{matplotlib} and \href{http://stanford.edu/~mwaskom/software/seaborn/}{seaborn} for plotting. Non-mandatory dependencies include \href{http://cython.org/}{cython} (used to speed up the execution of a few core functions)
and \href{http://ipython.org/notebook.html}{Jupyter Notebook} that can be used
to execute the tutorials.
A complete list of dependencies can be found in the 
\href{http://fretbursts.readthedocs.org/en/latest/getting_started.html}{Getting Started}
section in the FRETBursts documentation.

\subsection{Loading FRETBursts}
\label{sec:import}
In this article, we assume that FRETBursts is loaded using the following 
import line:

\begin{verbatim}
from fretbursts import *
\end{verbatim}

After this import, all the FRETbursts functions and objects will be accessible.
The previous line will also import \verb|numpy| and \verb|matplotlib.pyplot|
respectively as \verb|np| and \verb|plt|.

The previous import is also used in the FRETBursts tutorials.