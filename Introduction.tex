\section{Introduction}

\subsection{smFRET and burst analysis}

FRETBursts is an open source python software for burst analysis of confocal 
single-molecule Förster Resonance Energy Transfer (smFRET) data. The software can analyze smFRET measurements
with one or multiple excitation spots~\cite{Ingargiola_2013}. The supported 
excitation schemes include single lasers, Alternating Laser EXcitation (ALEX) with either two CW lasers (µs-ALEX \cite{Kapanidis_2005}) 
or two interleaved pulsed lasers (ns-ALEX \cite{16287971} or Pulsed-Interleaved Excitation (PIE) \cite{M_ller_2005}). 
FRETBursts is hosted and and openly developed~\cite{Prli__2012} on GitHub, where
issues, feature requests and contributions can be reported.

To minimize the chance of bugs and erroneous results, FRETBursts is developed
following modern software engineering techniques such 
as unit and regression tests. 

The preferential way to execute FRETBursts is by running one of the tutorials which are in the form of \href{http://ipython.org/notebook.html}{Jypyter notebooks}.
Jupyter (formerly IPython) notebooks are documents, 
accessed through a web browser, that contain both code and 
rich text (including equations, hyperlinks, figures, etc...).
FRETBursts tutorials are notebook that can be re-executed,
modified or used to process new data files with minimal modifications.
The "notebook workflow"\cite{Shen_2014} not only facilitates 
the description of the analysis (by integrating the code
in a rich document)
but also greatly enhance its reproducibility by storing an execution trail
that includes software versions, input files, parameters, commands and all
the analysis results (text, figures, tables, etc...).

Given the modality of execution, running FRETBurst does not require
any prior python knowledge. The user only needs familiarize with the
notebook graphical environment to be able to navigate and run the notebooks.
The list of FRETBursts notebooks can be found in the 
\verb|FRETBurts_notebooks| repository on GitHub.

Although running the software is technically simple, understanding the smFRET 
burst analysis requires several concepts and definitions.
This paper provides a broad overview of FRETBursts, introducing all the
the fundamental concepts and definitions needed to fully understand the 
smFRET burst analysis. 

The paper is structured as follows. 
In the next subsections mention how to install~(\ref{sec:install}) and 
load~(\ref{sec:import}) FRETBursts.
In section~\ref{sec:concepts} we
introduce a few preliminary concepts and some specific terminology needed 
to understand the smFRET burst analysis. 
In section~\ref{sec:analysis} we detail the execution the several steps involved
in smFRET burst analysis: data loading (section~\ref{sec:dataload}), defining 
excitation alternation periods (section~\ref{sec:alternation}), background 
correction (section~\ref{sec:bg_calc}), burst search (section~\ref{sec:burstsearch}), 
burst filtering (section~\ref{sec:burstsel}) and FRET fitting (section~\ref{sec:fretfit}).
The aim is providing enough information to understand the specificities of 
the different algorithms and to be able to adapt the analysis to new situations.
Finally, in section~\ref{sec:conclusions}, we summarize what we believe to be
the strengths of FRETBursts software.

In addition to this paper, we refer the interested readers to the 
\href{http://nbviewer.ipython.org/github/tritemio/FRETBursts_notebooks/blob/master/notebooks/FRETBursts\%20-\%20us-ALEX\%20smFRET\%20burst\%20analysis.ipynb}{FRETBursts µs-ALEX tutorial} 
for a single complete example of µs-ALEX data analysis and to the
\href{http://fretbursts.readthedocs.org/}{FRETBursts Reference Documentation}
for a more in-depth description of the software.

\subsection{Installing FRETBursts}
\label{sec:install}
Installing FRETBursts is analogous to installing any standard python
package. The python interpreter needs to be installed, preferentially
through a scientific python distribution which conveniently include 
all the major scientific libraries.

For new python users we suggest installing 
\href{https://store.continuum.io/cshop/anaconda/}{Continuum Anaconda}
python distribution and then install (or update) FRETBursts with the command:

\begin{verbatim}
conda install fretbursts -c tritemio
\end{verbatim}

For more information on different installation methods refer to the 
\href{http://fretbursts.readthedocs.org/en/latest/getting_started.html}{Getting Started}
section in the FRETBursts documentation.

FRETBursts depends on a standard set of scientific 
python libraries, namely, \href{http://www.numpy.org/}{numpy}/\href{http://www.scipy.org/}{scipy} for numerical computations, 
\href{http://matplotlib.org/}{matplotlib} and \href{http://stanford.edu/~mwaskom/software/seaborn/}{seaborn} for plotting. Non-mandatory dependencies include \href{http://cython.org/}{cython} (used to speed up the execution of a few core functions)
and \href{http://ipython.org/notebook.html}{Jupyter Notebook} that can be used
to execute the tutorials.
A complete list of dependencies can be found in the 
\href{http://fretbursts.readthedocs.org/en/latest/getting_started.html}{Getting Started}
section in the FRETBursts documentation.
