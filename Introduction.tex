\section{Introduction}

\subsection{smFRET and burst analysis}

FRETBursts is a python package for burst analysis of confocal single-molecule
FRET (smFRET) data.

In this paper we provide an broad overview of the FRETBursts software. The 
paper is structured as follows. In section~\ref{sec:concepts} we
introduce some basic concepts and specific terminology.
In section~\ref{sec:analysis} we illustrate how to execute the several steps involved
in smFRET burst analysis: background correction (section~\ref{sec:bg_calc}), burst search
(section~\ref{sec:burstsearch}), burst filtering (section~\ref{sec:burstsel}), 
FRET fitting (section~\ref{sec:fretfit}). For each step different approached and options 
are discussed.

In addition to this paper, we refer the interested readers to the 
\href{http://nbviewer.ipython.org/github/tritemio/FRETBursts_notebooks/blob/master/notebooks/FRETBursts\%20-\%20us-ALEX\%20smFRET\%20burst\%20analysis.ipynb}{FRETBursts µs-ALEX tutorial} 
for a single complete example of µs-ALEX data analysis and to the
\href{http://fretbursts.readthedocs.org/}{FRETBursts Reference Documentation}
for a detailed description of each function.

\subsection{Installing FRETBursts}
FRETBursts is a standard python package that requires the "scipy stack", a set
of core scientific python packages.
The "scipy stack" is easily installed through a free scientific python
distribution such as Continuum Anaconda, although some users may prefer another
distribution or a manual installation.

FRETBursts can be installed through the standard python package manager (PIP)
with 
the command \texttt{pip install fretbursts}. Alternatively the latest
development version can be installed from GitHub.
For more information on different installation methods see the
\href{http://fretbursts.readthedocs.org/en/latest/installation.html}{FRETBursts
documentation}.

\subsection{Executing FRETBursts}
In general, we suggest to import FRETbursts with the expression:

\begin{verbatim}
>>> import fretbursts as fb
\end{verbatim}

that will make available all the FRETBursts functions with a concise \verb|fb.|
prefix. In this article, however, we assume that FRETBursts is imported with the
shortcut form:

\begin{verbatim}
>>> from fretbursts import *
\end{verbatim}

that allows to skip the \verb|fb.| prefix and also imports some common numeric
libraries (numpy and matplotlib.pyplot imported as \verb|np| and \verb|plt|
respectively).

Furthermore we encourage using FRETBursts through the IPython Notebook
environment. IPython notebooks are documents that contains both code and
rich text and multimedia content.
The "notebook workflow"\cite{Shen_2014} has the advantage of automatically
recording not only the all analysis commands and parameters but also the
data file names, software versions and the full output 
(figures, tables, etc...) in a single document that can be interactively 
modified and re-executed.

All the FRETBursts tutorials are 
ipython notebooks and, indeed, a quick way to start a new analysis is copying 
a pre-existing FRETBursts notebook and modifying it.
