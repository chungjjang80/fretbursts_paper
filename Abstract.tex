
Single-molecule FRET experiments allows to probe the distance between two fluorescent labels (\textit{donor} and \textit{acceptor}) on the 3-10nm range. In the commonly employed confocal geometry, molecules are free to diffuse in a solution. When a molecule crosses the excitation volume it emits a burst of photons that can be detected by single-photon detectors (SPADs). In a nutshell, assessing the relative intensity of the donor and acceptor signal in each bursts it allows to infer their distance. smFRET experiments permit to study binding-unbunding processs or conformational changes of biomolecules and have proven to be an unvaluable tools in studying fundamental cellular processes \cite{Kapanidis_2006}.

The data analysis involves processing the photon timestamps in (at least) two spectral channels (donor and acceptor). 
Although smFRET experiments are routinely performed by many labs around the world, no standard software exists. FRETBursts represent the first complete opensource software for smFRET data analysis implementing most of the state-of-the-art algorithms. 
