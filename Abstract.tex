
Single-molecule FRET experiments allows to probe the distance between two fluorescent labels (\textit{donor} and \textit{acceptor}) on the 3-10nm range. In the commonly employed confocal geometry, molecules are free to diffuse in a solution. When a molecule crosses the excitation volume it emits a burst of photons that can be detected by single-photon detectors (SPADs). In a nutshell, assessing the relative intensity of the donor and acceptor signal in each bursts it allows to infer their distance. smFRET experiments permit to study binding-unbinding processes or conformational changes of biomolecules and have proven to be an unvaluable tool in studying many cellular processes \cite{Kapanidis_2006}.

Analyzing smFRET experiments involves identifing the single-molecule bursts in a continuous stream of photon timestamps, estimating the background and other correction factors, filtering and extracting the corrected FRET efficiencies for each sub-populations in the sample. Although smFRET experiments are routinely performed by many labs around the world, no standard or reference software exists. FRETBursts software described in this paper represents the first opensource implementation of all the common state-of-the-art algorithms for smFRET data analysis. We follow the highest standard in software development to ensure that the source is easy to read, well documented and thoroghly tested. Moreover, in an effort to lower the barriers to computational reproducibility, we embrace a modern workflow based on ipython notebooks that allows to capture the whole process from raw data to figures within a single document.
