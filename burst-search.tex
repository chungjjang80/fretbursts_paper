\subsection{Burst Search}
\label{sec:burstsearch}

\subsubsection{Burst Search in FRETBursts}
\label{sec:burstsearch_code}

Following background estimation, burst search is the next step of
the analysis.
In FRETBursts, a standard burst search on a single photon stream
(see section~\ref{sec:burstsearch_intro}) is performed by calling the
\href{http://fretbursts.readthedocs.org/en/latest/data\_class.html#fretbursts.burstlib.Data.burst\_search}{\texttt{burst\_search} method}.
For example, the following command:

\begin{lstlisting}
d.burst_search(F=6, m=10, ph_sel=Ph_sel('all'))
\end{lstlisting}

performs a burst search on all photons
(\verb|ph_sel=Ph_sel('all')|), with a minimum rate 6 times larger than the
background rate (\verb|F=6|) and using 10 consecutive photons to compute the
local rate (\verb|m=10|).
A different photon selection, threshold ($F$) or number of photons for rate
computation $m$ can be selected by passing a different value. These parameters
are generally a good starting point for smFRET analysis but can be adjusted in
specific cases.

Note that, in the previous burst search, no burst size selection was performed
(i.e. the minimum bursts size is effectively $m$).
An additional parameter $L$ can be passed to apply a threshold on the raw burst
size (before any correction).
It is recommended, however, to select bursts only after the background correction
is applied as shown in the next section~\ref{sec:burstsel}.

It might sometimes be useful to specify a fixed photon-rate threshold, instead
of a threshold depending on the background rate, as in the previous example. In
this case, instead of $F$, the argument \verb|min_rate_cps| can be used to
specify the threshold (in Hz). For example, a burst search with a 50~kHz
threshold can be performed as follows:

\begin{lstlisting}
d.burst_search(min_rate_cps=50e3, m=10, 
               ph_sel=Ph_sel('all'))
\end{lstlisting}

Finally, to perform a DCBS burst search (or in general an AND gate burst search,
see section~\ref{sec:burstsearch_intro}) the plugin function
\href{http://fretbursts.readthedocs.org/en/latest/plugins.html#fretbursts.burstlib\_ext.burst\_search\_and\_gate}{\texttt{burst\_search\_and\_gate}}
can be used as in the following example:

\begin{lstlisting}
d_dcbs = bext.burst_search_and_gate(d, F=6, m=10)
\end{lstlisting}

The last command puts the burst search results in a new copy the \verb|Data| variable \textit{d}
(the copy is here called \verb|d_dcbs|).
Since FRETBursts shares the arrays timestamps and detectors between
different copies of \verb|Data| objects, the memory usage is contained even when using 
several copies. 

\paragraph{Python details}
Note that, while \verb|.burst_search()| is a method of \verb|Data|,
\verb|burst_search_and_gate| is a function in the \verb|bext| module
taking a \verb|Data| object as a first argument and returning a new
\verb|Data| object.

The function \verb|burst_search_and_gate| accepts optional arguments,
\verb|ph_sel1| and \verb|ph_sel2|, whose default values correspond to the
classical DCBS photon stream selection (see section~\ref{sec:burstsearch_intro}).
These arguments can be specified to select different photon streams than in
a classical DCBS.

The \href{http://fretbursts.readthedocs.org/en/latest/plugins.html}{\texttt{bext} module} 
collects "plugin" functions that provides additional algorithms 
for processing \verb|Data| objects. 

\subsubsection{Correction coefficients}
\label{sec:corrcoeff}

In µs-ALEX there are 3 important correction parameters: γ-factor, donor spectral
leakage into the acceptor channel and acceptor direct excitation by the donor excitation
laser~\cite{Lee_2005}.
These corrections can be applied by simply assigning to the respective \verb|Data| attributes:

\begin{lstlisting}
d.gamma = 0.85
d.leakage = 0.04
d.dir_ex = 0.08
\end{lstlisting}

These attributes can be assigned either before or after the burst search. In the
latter case, existing burst data is automatically updated using the new
correction parameters.

These correction factors can be used to display corrected FRET distributions.
However, in order to resolve the peak FRET efficiency value, we found that is more accurate to 
fit the FRET histogram without those corrections (i.e. background-corrected
proximity ratio). Next, we can use an algebraic formula to correct fitted peak
positions (see SI of~\cite{Lee_2005}) and obtain the corrected FRET efficiency.
FRETBursts implements the correction formulas for $E$ and $S$ in the functions
\verb|fretmath.correct_E_gamma_leak_dir| and \verb|fretmath.correct_S|
(\href{http://fretbursts.readthedocs.org/en/latest/fretmath.html}{documentation}). 
A complete derivation of all these correction formulas
(and their inverse) has been posted as \href{http://nbviewer.jupyter.org/github/tritemio/notebooks/blob/master/Derivation%20of%20FRET%20and%20S%20correction%20formulas.ipynb}{notebook}.
