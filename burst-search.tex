\subsection{Burst search}
\label{sec:burstsearch}

\subsubsection{Burst search in FRETBursts}
\label{sec:burstsearch_code}

After background estimation, the burst search is the next fundamental step of
the analysis.
In FRETBursts, the standard burst search on a single photon stream (section~\ref{sec:burstsearch_intro}}) is performed calling the
\href{http://fretbursts.readthedocs.org/en/latest/data\_class.html#fretbursts.burstlib.Data.burst\_search}{\texttt{burst\_search} method}. For example the following command:

\begin{verbatim}
d.burst_search(F=6, m=10, ph_sel=Ph_sel('all'))
\end{verbatim}

performs a burst search on all photons
(\verb|ph_sel=Ph_sel('all')|), with a minimum rate 6 times larger than the
background rate (\verb|F=6|) and using 10 consecutive photons to compute the
local rate (\verb|m=10|).
A different photon selection, threshold ($F$) or number of photons for rate
computation $m$ can be selected by passing a different value. These parameters
are generally a good starting point for smFRET analysis but can be tweaked in
specific cases.

Note that, in the previous burst search, no burst size selection is performed
(i.e. the minimum bursts size is $m$). 
An additional parameter $L$ can be passed to apply a threshold on the raw burst
size (before any correction). 
We, however, strongly encourage performing an more accurate burst size selection as shown in
the next section~\ref{sec:burstsel}.

In µs-ALEX there are 3 important correction parameters: gamma factor, donor spectral
leakage into the acceptor channel and acceptor direct excitation~\cite{Lee_2005}. 
These corrections can be applied by simply setting the respective \textit{Data} attributes:

\begin{verbatim}
d.gamma = 0.85
d.leakage = 0.04
d.dir_ex = 0.08
\end{verbatim}

These attributes can be assigned either before or after the burst search. In the
latter case, the burst data is
automatically updated using the newly assigned correction values.

Sometimes it may be useful to specify a fixed threshold, instead 
of a threshold derived from the background rate like in the previous example. In
this case, instead of $F$ we can use the argument \verb|min_rate_cps| that
specifies a threshold in Hertz. For example, a burst search with a 50~kHz
threshold can be performed as follows:

\begin{verbatim}
d.burst_search(min_rate_cps=50e3, m=10, ph_sel=Ph_sel('all'))
\end{verbatim}

Finally, to perform a DCBS burst search (or in general an AND-gate burst search,
see section~\ref{sec:burstsearch_intro}) the plugin function 
\href{http://fretbursts.readthedocs.org/en/latest/plugins.html#fretbursts.burstlib\_ext.burst\_search\_and\_gate}{\texttt{burst\_search\_and\_gate}}
can be used like in the following example:

\begin{verbatim}
d_dcbs = bext.burst_search_and_gate(d, F=6, m=10)
\end{verbatim}

Note that in this case a copy of the \verb|Data| variable is returned (here called \verb|d_dcbp|)
containing the results of the new burst search. In order to
save RAM, FRETBursts shares the read-only arrays timestamps and detectors between
different copies of a \verb|Data| object (for example \verb|d| and
\verb|d_dcbs|), while all the other data (including background and burst data)
is copied. 

The function \verb|burst_search_and_gate| provides additional arguments
\verb|ph_sel1| and \verb|ph_sel2| whose defaults values correspond to the 
classical DCBS photon stream selection (see section~\ref{sec:burstsearch_intro}).
These arguments can be specified to select different photon streams than in 
a classical DCBS.
