\section{Architecture and concepts}
\label{sec:concepts}

In this section we introduce some general concepts and naming conventions related
to the smFRET burst analysis in FRETBursts.

\subsection{Photon streams}
\label{sec:ph_streams}

The fundamental data at the core of smFRET experiments is the array of photon
arrival timestamps, with a temporal resolution set by the acquisition hardware,
ranging from below nanoseconds to a few tens of nanoseconds.
In single-spot
measurements, all timestamps are stored in a single array. In multi-spot
measurements~\cite{Ingargiola_2013}, there are as many timestamps arrays
as excitation spots.

Each array contains timestamps from both donor (D) and acceptor (A) channels.
In ALEX measurements~\cite{Lee_2005}, we can further differentiate between
photons emitted during D and A excitation periods. In FRETBursts the different
selections of photons/timestamps are called "photon streams" and they are
specified with a
\href{http://fretbursts.readthedocs.org/en/latest/ph_sel.html}{\texttt{Ph\_sel}
object} . In non-ALEX smFRET data there are 3 photon streams
(table~\ref{tab:ph_sel_smfret}), while in ALEX data we have 5 base photon
streams (table~\ref{tab:ph_sel_alex}).

The
\href{http://fretbursts.readthedocs.org/en/latest/ph_sel.html}{\texttt{Ph\_sel}
class} allows the expression of any combination of photon streams.
For example, in ALEX measurements, the D-emission during A-excitation stream is
usually excluded because it does not contain any useful signal~\cite{Lee_2005}.
To indicate all but the photons in this photon stream we write
\verb|Ph_sel(Dex='DAem', Aex='Aem')|, which indicates \textit{selection of donor
and acceptor photons (DAem) during donor excitation (Dex) and only acceptor
photons (Aem) during acceptor excitation (Aex)}.

\begin{table}
\begin{tabular}{l|l}
  Photon selection  & code \\
  \hline
  All-photons       & \verb|Ph_sel('all')|\\
  D-emission    & \verb|Ph_sel(Dex='Dem')|\\
  A-emission & \verb|Ph_sel(Dex='Aem')|\\
\end{tabular}
\caption{\label{tab:ph_sel_smfret}Photon selection syntax (non-ALEX)}
\end{table}

\begin{table}
\begin{tabular}{l|l}
  Photon selection  & code \\
  \hline
  All-photons & \verb|Ph_sel('all')|\\
  D-emission during D-excitation & \verb|Ph_sel(Dex='Dem')|\\
  A-emission during D-excitation & \verb|Ph_sel(Dex='Aem')|\\
  D-emission during A-excitation & \verb|Ph_sel(Aex='Dem')|\\
  A-emission during A-excitation & \verb|Ph_sel(Aex='Aem')|\\
\end{tabular}
\caption{\label{tab:ph_sel_alex}Photon selection syntax (ALEX)}
\end{table}

\subsection{Background definitions}
\label{sec:bg_intro}

An estimation of the background rates is needed both to select a proper threshold for
burst search and to correct the raw burst counts by subtracting the background counts.

The recorded stream of timestamps is the result of two processes: one characterized
by a high count rate, due to fluorescence photons of single molecules crossing the
excitation volume, and another one characterized by a lower count rate due to “background
counts” originating from the detectors dark counts, out of focus molecules
and sample scattering and/or auto-fluorescence\cite{Gopich_2008}.
The signature of those two processes can be
observed in the distribution of timestamp delays (i.e. the waiting times
between two subsequent timestamps) as illustrated in Figure~\ref{fig:bgdist}.
The “tail” of the distribution (a straight line in semi-log scale) corresponds
to exponentially-distributed delays, indicating that those counts are generated by a
\href{http://en.wikipedia.org/wiki/Poisson_process}{Poisson process}. At short
timescales, the distribution departs from exponential behavior due to the contribution
of the higher rate process of single molecules traversing the excitation volume.
To estimate the background rate, (i.e. the exponential time constant)
it is necessary to define a delay threshold, above which the distribution
can be considered exponential.
Next a fitting method, for example the Maximum
Likelihood Estimation (MLE) or a curve fit of the histogram via non-linear
least squares (NLSQ) must be selected.

It is advisable to check the background at different time points
throughout the measurements in order to track possible variations.
Experimentally, we found that when the background is not constant,
it usually varies
on time scales of tens of seconds (see figure~\ref{fig:bg_timetrace}).
FRETBursts splits the data in uniform time
slices called \textit{background periods} and computes the background rates for
each of these slices (see section~\ref{sec:bg_calc}).
Note that the the same splitting in background periods is used during
burst search to compute a background-dependent
threshold and to apply the burst correction (section~\ref{sec:burstsearch}).

\subsection{The \texttt{Data} class}
\label{sec:data_intro}

The
\href{http://fretbursts.readthedocs.org/en/latest/data_class.html}{\texttt{Data}
class} is the fundamental data container in FRETBursts. It contains the
measurement data and provides several methods for data analysis (background
estimation, burst search, etc...). It also stores all analysis results
(bursts data, estimated parameters).

As an example, there are 3 important ``burst counts'' fields containing  
the number of photon detected in donor or acceptor channel
during donor or acceptor excitation:

\begin{itemize}
\item \verb|nd|: number of photons detected by the donor channel
(during donor excitation period, if ALEX)
\item \verb|na|: number of photons detected by the acceptor channel
(during donor excitation period, if ALEX)
\item \verb|naa|: number of photons detected by the acceptor channel
during acceptor excitation period (present only in ALEX measurements)
\end{itemize}

These fields are background-corrected by default. Furthermore,
\verb|na| is corrected for leakage and direct excitation if the
relative coefficients are specified (by default they are 0).
There is also a field named \verb|nda| containing donor photon during 
acceptor excitation, which is only due to background.

\paragraph{Python details}

Most arrays in \texttt{Data} are contained in lists with lengths equal to the
number of excitation spots. This means that, in
single-spot measurements, to access an array of burst-data
we always have to specify the index 0, for example \verb|Data.nd[0]|.
\verb|Data| implements a shortcut syntax to access the first element of a list
with an underscore, so we can type equivalently use
\verb|Data.nd_| instead of \verb|Data.nd[0]|.

\subsection{Introduction to burst search}
\label{sec:burstsearch_intro}

Identifying single-molecule bursts in the stream of photons is
one of the most crucial steps in the analysis of freely-diffusing single-molecule FRET data.
The widely used "sliding window" algorithm, introduced by the Seidel group in 1998
(\cite{Eggeling_1998}, \cite{Fries_1998}), involves searching for
$m$ consecutive photons detected during a period shorter than
$\Delta t$. In other words, bursts are regions of the photon stream where the
local rate (computed using $m$ photons) is above a minimal rate chosen as a
threshold. Eggeling did not provide any criteria on how to choose the rate
threshold and the number of photons $m$ and as therefore it has become a common
practice to manually adjust those parameters for each specific measurement.

A more general approach consists in taking into account the background rate of
the specific measurements and in choosing a rate threshold that is $F$ times
larger than the background rate. This approach assures that all the resulting bursts
have a signal-to-background ratio (SBR) larger than
$(F-1)$~\cite{Michalet_2012}. A consistent criterion for choosing the threshold is
particularly important when comparing different measurements with different background
rates, when the background significantly varies during measurements or in
multi-spot measurements where each spot has a different background rate.

A second important aspect of burst search is the choice of photon stream used
to perform the search.
In most cases, for instance when identifying FRET populations,
the burst search should use all photons. In some other cases, when focusing on
donor-only or acceptor only populations, it is better to perform the search using
only donor or acceptor signal.
In order to handle the general case and to provide flexibility,
FRETBursts allows to perform the burst search on arbitrary selections of photons.
(see section~\ref{sec:ph_streams} for more info on photon stream definitions).

Additionally, Nir~\textit{et al.}~\cite{Nir_2006} proposed a Dual-Channel Burst
Search (DCBS), which can help mitigating artifacts due to
photo-physical effects such as blinking. In this case a search is performed
independently on two photon streams and bursts are marked only when both photon
streams exhibit a rate higher than the threshold,
implementing an AND-gate logic.
Usually, the term DCBS refers to a burst search where the two photon streams
are (1) all photons
during donor excitation (\verb|Ph_sel(Dex='DAem')|) and (2) acceptor channel photons
during acceptor
excitation (\verb|Ph_sel(Aex='Aem')|).

After each burst search, it is useful to select
bursts having a minimum number of photons (burst size). In the most
basic form, this selection can be performed during burst search by discarding
bursts with size smaller than a threshold $L$, as originally proposed by
Eggeling~\textit{et al.}~\cite{Eggeling_1998}.
This method, however, neglects the effect
of background and gamma factor on the burst size and can lead to a selection
bias of certain channels and/or sub-populations.
For this reason we encourage performing a burst size selection after background
correction, possibly taking into account the gamma factor, as discussed in
sections~\ref{sec:burstsizeweights} and~\ref{sec:burstsel}.

\subsection{Gamma-corrected burst sizes and weights}
\label{sec:burstsizeweights}

The number of photon detected during a burst is commonly called the ``burst size''.
Bursts sizes are usually computed using either all photons, or photons detected 
during donor excitation period. To compute burst size, FRETBursts uses 
one of the following formulas:

\begin{equation}
\label{eq:burstsize_dex}
n_t = n_a + \gamma\,n_d 
\end{equation}

\begin{equation}
\label{eq:burstsize_allph}
n_t = n_a + \gamma\,n_d + n_{aa}
\end{equation}

The former includes  only photons during donor excitation periods, 
while the latter includes all photons.
The inclusion of $\gamma$ factor allows to obtain a ``fair threshold''
and to correct bias when comparing the number of bursts across several sub-populations.

The results of burst search is a distribution of burst sizes that approximately 
follows an exponential distribution.
Bursts with large size (which contain most of the information)
are much less frequent than bursts with smaller sizes. For this reason, it is 
important to select burst sizes larger than a threshold in order
to properly characterize FRET populations (see section~\ref{sec:burstsel}). 

Selecting bursts by size is a critically important step.
A too low threshold will broaden the FRET populations and introduce
artifacts (spurious peaks and patterns) due to the majority of bursts
having E and S computed from ratios of small integers. 
Conversely, a too high threshold will result in a lower number of bursts 
and possibly poor statistics in representing FRET populations.
Additionally, when selecting bursts (see section~\ref{sec:burstsel}), 
it is important to use $\gamma$-corrected burst sizes,
in order to avoid under-representing some FRET sub-populations
due to different quantum yields between donor and acceptor dyes and/or 
different photon detection efficiencies of donor and acceptor emission.

A simple way to mitigate the dependence on the burst size threshold is
weight bursts according to their size (i.e. their information content)
so that the biggest bursts will have the highest weights.
The weighting can be used to build weighted histograms and Kernel Density 
Estimation (KDE) plots. When using weights, the choice of a particular 
burst size threshold affects less the shape of the burst distribution,
so lower thresholds can be used (better statistics) without broadening 
the peaks of sub-populations (better population identification).

\subsubsection{Python details}
FRETBursts has the option to weight bursts using $\gamma$-corrected 
burst sizes and to include the acceptor excitation photons \verb|naa|.
A weight proportional to the burst size is applied by passing the argument 
\verb|weights='size'| to histogram or KDE plot functions. The \verb|weights|
keyword can be also passed to fitting functions in order to fit 
the weighted E or S distributions (see section~\ref{sec:fretfit}).
Several other weighting functions (for example quadratical) are listed in the
\href{http://fretbursts.readthedocs.org/en/latest/fret_fit.html#fretbursts.fret_fit.get_weights}{\texttt{fret\_fit.get\_weights} documentation}.

\subsection{Plotting "Data"}
\label{sec:plotting}

FRETBursts uses 
matplotlib~\cite{matplotlib}
and seaborn~\cite{seaborn} 
to provide a wide range of
\href{http://fretbursts.readthedocs.org/en/latest/plots.html}{built-in plot functions}
for \verb|Data| objects.
The plot syntax is the same for both single and multi-spot measurements.
The majority of plot commands are called through the wrapper function
\verb|dplot|, for example to plot a timetrace of the photon data, type:

\begin{lstlisting}
dplot(d, timetrace)
\end{lstlisting}

The function \verb|dplot| is the generic plot function which creates figure
and handles details common to all the plotting functions (for instance the title).
\verb|d| is the \verb|Data| variable and \verb|timetrace| is the actual plot
function which operates on a single channel. In multi-spot measurements
\verb|dplot| creates one subplot for each spot and calls \verb|timetrace| for
each channel.

All built-in plot functions which can be passed to
\verb|dplot| are defined in the
\href{http://fretbursts.readthedocs.org/en/latest/plots.html}{\texttt{burst\_plot} module}.

\paragraph{Python details}

When FRETbursts is imported, all the plot functions are also imported.
To facilitate finding the plot functions through auto-completion,
their names start with a standard prefix indicating the
plot type. The prefixes are: \verb|timetrace| for binned timetraces
of photon data, \verb|ratetrace| for rates of photons as a function of time (non
binnned), \verb|hist| for functions plotting histograms and \verb|scatter| for
scatter plots.

Additional plots can be easily created directly with matplotlib.

By default, in order to speed-up batch processing, FRETBursts notebooks display plots 
as static images using the \textit{inline} matplotlib backend. 
User can choose switch to interactive figures by using an interactive backend. 
For example \verb|%matplotlib notebook| 
provides interactive figures inside the browser, 
and \verb|%matplotlib qt| 
for interactive figures in a new window (using the QT4 graphical library).

A few plot functions such as \verb|timetrace| and \verb|hist2d_alex| have interactive features 
which require the QT4 backend. As an example, after switching to the QT4 backend
the following command:

\begin{lstlisting}
dplot(d, timetrace, scroll=True, bursts=True)
\end{lstlisting}

\noindent
will open a new window with a timetrace plot with overlay of bursts, and an horizontal scroll-bar for quick
"scrolling" throughout the measurement. The user can click on a burst to have the corresponding burst info 
printed in the notebooks.
Similarly, calling the \verb|hist2d_alex| function with the QT4 backend allows
selecting an area on the E-S histogram using the mouse.

\begin{lstlisting}
dplot(ds, hist2d_alex, gui_sel=True)
\end{lstlisting}

The values that identify the region are printed in the notebook and can be used 
to select bursts inside the region with the function \verb|select_bursts.ES| (see
section~\ref{sec:burstsel}).
