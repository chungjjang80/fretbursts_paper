\subsection{Background Estimation}
\label{sec:bg_calc}

The first step of smFRET analysis involves estimating background rates.
For example, to compute the background every 30 s, using a minimal inter-photon
delay threshold of 2~ms for all the photon, we use:

\begin{lstlisting}
d.calc_bg(bg.exp_fit, time_s=30, tail_min_us=2000)
\end{lstlisting}

The first argument (\verb|bg.exp_fit|) is the underlying function used to fit the
background in each period and for each photon stream (see section~\ref{sec:bg_intro}).
The function
\verb|bg.exp_fit| estimates the background using a maximum likelihood estimation
(MLE) of the delays distribution. Additional fitting functions are available in
\verb|bg| namespace 
(i.e. the \verb|background| module, \href{http://fretbursts.readthedocs.org/en/latest/background.html}
{link}). The second argument, \verb|time_s|, is the
\textit{background period} (section~\ref{sec:bg_intro}) and the third, \verb|tail_min_us|,
is the inter-photon delay threshold above which the distribution is assumed exponential.
It is possible to use different thresholds for each photon stream, passing a
tuple (i.e. a comma-separated list of values, \href{https://docs.python.org/3.5/tutorial/datastructures.html#tuples-and-sequences}{link}) instead of a scalar.
Finally, it is possible to use a heuristic estimation of the threshold using
\verb|tail_min_us='auto'|. For more details refer to the \verb|calc_bg| documentation
(\href{http://fretbursts.readthedocs.org/en/latest/data_class.html#fretbursts.burstlib.Data.calc_bg}{link}).

FRETBursts provides are two kind of plots to represent the background. One is the histograms
of inter-photon delays compared to the fitted exponential distribution reported in 
figure~\ref{fig:bg_dist_all}) (see section~\ref{sec:bg_intro} for details on the inter-photon distribution). 
This plot is performed with the command:

\begin{lstlisting}
dplot(d, hist_bg, bp=0)
\end{lstlisting}

The argument \verb|bp| is an integer specifying the background period to be plotted.
When not specified the default is 0, i.e. the first period.
Figure~\ref{fig:bg_dist_all} allows to quickly identify pathological cases when the 
background fitting procedure returns unreasonable values. 

The second background-related plot is a timetrace of background rates, 
as shown in figure~\ref{fig:bg_timetrace}. This plot allows to monitor background changes
taking place during the measurement and is obtained with the command:

\begin{lstlisting}
dplot(d, timetrace_bg)
\end{lstlisting}

Normally, samples should have a constant background as a function of time
like in figure~\ref{fig:bg_timetrace}(a). However, oftentimes, non-ideal
experimental conditions can yield a time-varying background, as shown in
figure~\ref{fig:bg_timetrace}(b).
For example, when the sample is not sealed in an observation chamber,
evaporation can induce background variations (typically increasing)
as a function of time. Additionally,
cover-glass impurities can contribute to the background even when focusing 
deep into the sample (10μm or more).
These impurities tend to bleach on timescales of minutes resulting in
background variations during the course of the measurement.

\paragraph{Python details} For an ALEX measurement, the tuple passed to
\verb|tail_min_us| to define the thresholds, is required to have have 
5 values corresponding the 5 photon streams. 
The order of the photon streams can be obtained from
the \verb|Data.ph_streams| attribute (i.e. \verb|d.ph_streams| in our example).
The estimated background rates are stored in the \verb|Data| attributes
\verb|bg_dd|, \verb|bg_ad| and \verb|bg_aa|, corresponding to the photon
streams \verb|Ph_sel(Dex='Dem')|, \verb|Ph_sel(Dex='Aem')| and \verb|Ph_sel(Aex='Aem')|
respectively. These attributes are lists of arrays (one array per excitation spot).
The arrays contain the estimated background rates in the different background periods.

\subsubsection{Error Metrics and Optimal Threshold}

The functions used to fit the background provide also a goodness-of-fit estimator 
computed on the basis of the empirical distribution function (EDF)~\cite{Stephens1974,Parr1980}. 
The ``distance'' between the EDF and the theoretical (i.e. exponential) cumulative distribution
represents and indicator of the quality of fit.
Two different distance metrics can be returned by the background fitting functions.
The first is the Kolgomorov-Smirnov statistics, which uses the maximum of the difference 
between the EDF and the theoretical distribution. The second is the Cramér von Mises
statistics corresponding to the integral of the squared residuals
(see the code for more details,
\href{https://github.com/tritemio/FRETBursts/blob/master/fretbursts/background.py#L41}{link}).

In principle, the optimal inter-photon delay threshold will minimize
the error metric. This approach is implemented by the function \verb|calc_bg_brute|
(\href{http://fretbursts.readthedocs.org/en/latest/plugins.html#fretbursts.burstlib\_ext.calc\_bg\_brute}{link}) which performs a brute-force search in order to find the optimal threshold.
This level of sophistication in estimating the background rates is not
necessary under typical experimental conditions, as the difference between an optimal threshold
and a manually (or heuristically) chosen one will be small if not negligible in most practical cases.
