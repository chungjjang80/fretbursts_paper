\section{Coding style, testing and bug disclosure}

FRETBursts has a well commented code-base with more that 35\% of source code
being comments (for comparison the average project registered on OpenHub has 25\%
of comments). To ease readability we follow most of the 
\href{https://www.python.org/dev/peps/pep-0008/}{PEP8 code style rules} and 
the highly-readable \href{http://sphinxcontrib-napoleon.readthedocs.org/}{napoleon format}
for the docstrings.

The reference documentation is built from the source files with \href{http://sphinx-doc.org/}{Sphinx} with
all the API documentation automatically generated from the docstrings.
On each commit the documentation is automatically built by 
\href{https://readthedocs.org/}{Read the Docs}
and available online after a few minutes.

Unit tests cover most of the core algorithms ensuring consistency and 
minimizing the probability of introducing bugs. The continuous integration
service \href{http://travis-ci.org}{Travis CI} is used to execute the full
test suite after each commit.
As a rule, whenever a bug is discovered, the  fix also includes a new test 
to ensure that the same bug cannot happen in the future.

In addition of the unit tests, a specialized 
\href{https://github.com/tritemio/FRETBursts/blob/master/notebooks/dev/tests/FRETBursts\%20-\%20Regression\%20tests.ipynb}{regression-test notebook}
that compares results between two versions of FRETBursts is
included in the source tree and periodically executed. Additionally
the tutorials itself are periodically executed to ensure that
no errors or regressions are introduced.

Finally, the full set of notebooks for the multi-spot paper analysis,
including notebooks for extensive µs-ALEX analysis are periodically
re-executed and results compared across executions to ensure that 
they agree within the floating point tolerances.

Code readability, rigorous development practice and extensive testing
are all efforts undertaken to minimize the presence of bugs.

We commit at publicly disclosing all the bugs that may result in erroneous results 
as soon as they are discovered or communicated to us.
