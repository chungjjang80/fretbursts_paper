\subsection{Population analysis}
\label{sec:fretfit}

The \verb|mfit| fitting module allows fitting burst populations with a model.
Typically, after bursts selection, E or S histograms are fitted to a model.

In general, models can be automatically built from any function.
Model parameters need to have an initial value (and optionally
constrains). 

FRETBursts includes a set of predefined models such as 1 to 3 Gaussian
peaks or 2-Gaussian plus ``bridge''.
Built-in models are created calling the a factory function 
(names starting with \verb|mfit.factory_|) which initialize the parameters
with initial values and constraints suitable for E and S histogram fit.
(see
\href{http://fretbursts.readthedocs.org/en/latest/mfit.html#model-factory-functions}{factory-functions documentation}).
Additionally models can be created from any generic function.

As an example, in order to fit the E histogram of bursts in \verb|ds| with two
Gaussian peaks, the following command can be used:

\begin{lstlisting}
bext.bursts_fitter(ds, 'E', binwidth=0.03,
                   model=mfit.factory_two_gaussians())
\end{lstlisting}

Here, \verb|ds| is the variable with the burst data (selected bursts),
\verb|'E'| is the name of the \textit{Data} field to fit, \verb|binwidth| is the bin
width of the histogram and \verb|model| is a pre-initialized model used for
fitting.

All fitting results are stored in the \textit{Data} variable (in the \verb|E_fitter| or
\verb|S_fitter| field).
In order to plot the FRET histogram and the fitted model, we pass the parameter
\verb|show_model=True| to \verb|hist_fret| function as follows
(see also section~\ref{sec:plotting}):

\begin{lstlisting}
dplot(ds, hist_fret, show_model=True)
\end{lstlisting}

For more examples on fitting bursts data and plotting results see the
\href{http://nbviewer.ipython.org/urls/raw.github.com/tritemio/FRETBursts_notebooks/master/notebooks/FRETBursts\%2520-\%2520us-ALEX\%2520smFRET\%2520burst\%2520analysis.ipynb}{us-ALEX notebook}.
Refer to the FRETBursts documentation for the different fitting and model options
(\href{http://fretbursts.readthedocs.org/en/latest/fit.html}{Fitting framework} and
\href{http://fretbursts.readthedocs.org/en/latest/plugins.html#fretbursts.burstlib\_ext.bursts\_fitter}{\texttt{bursts\_fitter} function})
and for details on plot customization (
\href{http://fretbursts.readthedocs.org/en/latest/plots.html#fretbursts.burst_plot.hist_fret}{\texttt{hist\_fret} plot function}).

\paragraph{Python details}
Under the hood, histogram fit in FRETBursts is performed using the
\href{http://lmfit.github.io/lmfit-py/}{lmfit} library~\cite{lmfit}.
Every fit model is, in fact, an \verb|lmfit.Model| object, and these object
can be created and modified by calling directly \verb|lmfit| functions.
Fitting an lmfit \verb|Model| returns an object with all the results.
FRETBursts appends the fit result of each channel in the list 
\verb|ds.E_fitter.fit_res|.
Example of defining and modifying models for fitting are provided in the FRETBursts tutorial. 
Interested users should also refer to the excellent lmfit documentation.

\subsubsection{Correction coefficients}

\begin{itemize}
\item Fit D-only and A-only population.
\item Fit gamma factor.
\end{itemize}


\subsubsection{Accurate FRET}

Apply corrections to the bursts vs apply corrections after the fit.


