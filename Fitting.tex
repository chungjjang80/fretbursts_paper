\subsection{Population analysis}
\label{sec:fretfit}

Typically, after bursts selection, E or S histograms are fitted to a model.
FRETBursts \verb|mfit| module allows fitting histograms of bursts quantities
(i.e. E or S) with arbitrary models. In this context, a model is an object 
specifying a function, the parameters varied during the fit
and optional contraints for these parameters. This concept of model
is taken from \textit{lmfit}~\cite{lmfit}, the underlying library used by
FRETBursts to perform the fits.

Models can be created from arbitrary functions. For convenience,
FRETBursts allows to use predefined models such as 1 to 3 Gaussian
peaks or 2-Gaussian plus ``bridge''.
Built-in models are created calling a corresponding factory function
(names starting with \verb|mfit.factory_|) which initializes the parameters
with values and constraints suitable for E and S histograms fits.
(see \textit{Factory Functions} documentation, 
\href{http://fretbursts.readthedocs.org/en/latest/mfit.html#model-factory-functions}{link}).

Continuing our example, in order to fit the E histogram of bursts in the
\verb|ds| variable with two Gaussian peaks, we use the following command:

\begin{lstlisting}
bext.bursts_fitter(ds, 'E', binwidth=0.03,
                   model=mfit.factory_two_gaussians())
\end{lstlisting}

Changing \verb|'E'| with \verb|'S'| will fit the S histogram instead.
The argument \verb|binwidth| specifies the histogram bin width and 
the argument \verb|model| takes pre-initialized model used to be used for
fitting.

All fitting results (including best fit values, uncertainties, etc...), 
are stored in the \verb|E_fitter| (or \verb|S_fitter|)
attributes of the \verb|Data| variable (here named \verb|ds|).
In order to plot the fitted model with the FRET histogram, we pass the parameter
\verb|show_model=True| to \verb|hist_fret| function as follows
(see also section~\ref{sec:plotting}):

\begin{lstlisting}
dplot(ds, hist_fret, show_model=True)
\end{lstlisting}

For more examples on fitting bursts data and plotting results see the
μs-ALEX notebook (\href{http://nbviewer.jupyter.org/github/tritemio/FRETBursts_notebooks/blob/master/notebooks/FRETBursts%20-%20us-ALEX%20smFRET%20burst%20analysis.ipynb}{link}).
the \textit{Fitting Framework} section of the documentation 
(\href{http://fretbursts.readthedocs.org/en/latest/fit.html}{link})
as well as the \verb|bursts_fitter| function documentation
(\href{http://fretbursts.readthedocs.org/en/latest/plugins.html#fretbursts.burstlib\_ext.bursts\_fitter}{link}).

\paragraph{Python details}
Under the hood, histogram fits in FRETBursts are performed using the
\href{http://lmfit.github.io/lmfit-py/}{lmfit} library~\cite{lmfit}.
Models returned by FRETBursts's factory functions (\verb|mfit.factory_*|) 
are \verb|lmfit.Model| objects (\href{https://lmfit.github.io/lmfit-py/model.html}{link}).
Custom models can be created calling \verb|lmfit.Model| directly.
When an \verb|lmfit.Model| is fitted, it returns a \verb|ModelResults| object 
(\href{https://lmfit.github.io/lmfit-py/model.html#the-modelresult-class}{link})
which contains all the information related to the fit (model, data, 
parameters with best values and uncertainties) and useful methods to operate on fit results. 
FRETBursts puts \verb|ModelResults| of each channel in the list 
\verb|ds.E_fitter.fit_res|.
As an example, to get the reduced $\chi^2$ value of the E histogram fit in a 
single-spot measurement \verb|d|, we use:

\begin{lstlisting}
d.E_fitter.fit_res[0].redchi
\end{lstlisting}

Other useful attributes are \verb|aic| and \verb|bic| which contain the 
Akaike information criterion (AIC) and the Bayes Information criterion (BIC)
quantities. AIC and BIC allow to compare different models and
to select the most approriate for the data at hand.

Example of defining and modifying models for fitting are provided in 
the μs-ALEX FRETBursts tutorial (\href{http://nbviewer.jupyter.org/github/tritemio/FRETBursts_notebooks/blob/master/notebooks/FRETBursts%20-%20us-ALEX%20smFRET%20burst%20analysis.ipynb}{link}). 
Interested users should also refer to the comprehensive lmfit's documentation
(\href{http://lmfit.github.io/lmfit-py/}{link}).
