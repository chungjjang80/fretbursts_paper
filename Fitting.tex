
\subsection{Population fit}

The fitting module \verb|mfit| allows to fit burst populations with a model. Typically, a bursts selection is perfomed first and aftewards the histograms of the $E$ or $S$ values of the selected bursts are fitted to a model. Under the hood the histogram fit is performed using the \href{http://lmfit.github.io/lmfit-py/}{lmfit} library, and, in fact, 
a fit model is an \verb|lmfit.Model| object.

In general a model can be automatically built from any function. Before fitting, however, all the model parameters need to have an initial value and (optionally) some constrains. FRETBursts includes a set of functions (starting with \verb|mfit.factory_|) that return pre-initialized models suitable for E and S histogram fitting. A set of commonly used functions, for example 1 or 2 gassian peaks with or without "bridge" function, are provided. These "factory" functions take optional arguments to change the initital value for the model parameters. For more info refer to the
\href{http://fretbursts.readthedocs.org/en/latest/mfit.html#model-factory-functions}{factory-functions documentation}.

For example, to fit the E histogram of bursts in \verb|ds| with two Gaussian peaks we execute:

\begin{verbatim}
bext.bursts_fitter(ds, 'E', binwidth=0.03, model=mfit.factory_two_gaussians())
\end{verbatim}

Here, \verb|ds| is the variable with the burst data (selected bursts), \verb|'E'| is the name of the Data field to fit, \verb|binwidth| is the bin width of the histogram and \verb|model| is a pre-initialized model used for fitting.

After fitting, all the fitting results are stored in the Data variable (in the example in the \verb|E\_fitter| field).
To plot the FRET histogram and the fitted model we run:

\begin{verbatim}
dplot(ds, hist_fret, show_model=True)
\end{verbatim}

For more example on fitting bursts data and plotting results see the \href{http://nbviewer.ipython.org/urls/raw.github.com/tritemio/FRETBursts_notebooks/master/notebooks/FRETBursts\%2520-\%2520us-ALEX\%2520smFRET\%2520burst\%2520analysis.ipynb}{us-ALEX notebook}.

For more info on defining custom model or on changing the constraints


\subsubsection{Correction coefficients}


\begin{itemize}
\item Fit D-only and A-only population.
\item Fit gamma factor.
\end{itemize}


\subsubsection{Accurate FRET}

Apply corrections to the bursts vs apply corrections after the fit.

