\subsection{Timestamps and burst data}
\label{sec:burststimes}

FRETBursts provides the infrastructure for exploring new analysis approaches.
Users can easily get timestamps (or selection masks) for any photon stream.
Core burst data (essentially start and stop times and indexes 
and derived quantities) are stored in special \verb|Bursts| objects 
(\href{documentation}{http://fretbursts.readthedocs.org/en/latest/burstsearch.html}).
These objects provide a simple and well tested interface (100 \% test coverage) 
to access and manipulate burst data. Bursts are created from a sequence of start/stop 
times and indexes, while all the other fields are automatically
computed. \verb|Bursts|'s methods allow to recompute indexes relative to a different photon
selection or recompute start and stop times relative to a new timestamps array.
Additional methods perform fusion of nearby bursts or intersection of two set
of bursts (functionality used by the dual-channel burst search).
In conclusion, \verb|Bursts| efficiently implements all the common operations performed 
with burst data, providing and easy-to-use interface and well tested algorithms. 
Leveraging \verb|Bursts| methods, users can implement new types of analysis without 
wasting time implementing (and debugging) standard manipulation routines.

\subsubsection{Python details}
Bursts objects store the start and stop times and indexes in a numpy array.
The other fields are computed on-fly using class properties, so they are always
up to date even if start and stop are modified. Iteration over bursts is
relatively fast, with performances similar to iterating through numpy rows.

