\subsection{Burst search}
\label{sec:burstsearch}

\subsubsection{Introduction to burst search}
\label{sec:burstsearch_intro}

After background estimation, the burst search is the next fundamental step of the analysis. The core "sliding window" algorithm, proposed by Eggeling~\textit{et al.} in 1998~\cite{Eggeling_1998}, involves searching for bursts of photons
in which $m$ consecutive photons are within a minimum time lag $\Delta t$. In other words, the bursts are the portions of the photon stream where the local a rate (computed using $m$ photons) is above a minimum rate chosen as a threshold. Eggeling did not provide any criteria on how to choose the rate threshold and the number of photons $m$ and as therefore it has become a common practice to manually tweak those parameters for each specific measurement. 

A more general approach consist in taking into account the background rate of the specific measurements and in choosing a rate threshold that is $F$ times larger than the background rate. This approach assures that the resulting bursts all have a single-to-background ratio (SBR) larger than $(F-1)$~\cite{Michalet_2012}. A consistent criterium to choose the threshold is very important when comparing different measurements with different background rates, when the background significantly changes during the measurements or in multi-spot measurements where each spot has a different background rates.

A second important aspect of burst search is which photon stream is processed. Usually, when identifying FRET populations, we want to apply the burst search to all the photons. Other times, when focusing on donor-only or acceptor only population, is better to use only the donor or acceptor signal. In general we want to be able to apply the burst search to an arbitrary selection of photons. In FRETBursts this can be achieved passing the appropriate \verb|Ph_sel| object to the burst search method (see section~\ref{sec:ph_streams} for more info on photon stream definitions).

Finally, Nir~\textit{et al.}~\cite{Nir_2006} proposed a dual-channel burst search (DCBS) that allows to mitigate artifacts due to photo-physical effects such as blinking. In this case a search is performer independently on two photon streams and bursts are marked only when both photon streams exhibit a rate higher than the threshold, 
implementing a kind of AND-gate logic. 
Usually, the term DCBS is refers to a burst search where the two photon streams are all the photons 
during donor excitation (\verb|Ph_sel(Dex='DAem')|) and acceptor channel photons during acceptor 
excitation (\verb|Ph_sel(Aex='Aem')|).

After the first level of burst search is performed it is important to select bursts according to their number of photons (burst size). In the most rudimentaly form this selection can be perfomed during burst search discarding all the bursts
with size lower that a threshold $L$. This method, however, neglects the effect of background and gamma factor on the burst size and can lead to a selection bias of certain channels or of certain sub-populations. 
For this reason we advocate performing a burst size selection after background correction and taking into account the gamma factor, as illustrated in section~\ref{sec:burstsel}.

\subsubsection{Burst search in FRETBursts}
\label{sec:burstsearch_code}

In FRETBursts the standard burst search is performed calling the \href{http://fretbursts.readthedocs.org/en/latest/data_class.html#fretbursts.burstlib.Data.burst_search_t}{\verb|burst_search_t| method}.

\begin{verbatim}
d.burst_search(F=6, m=10, ph_sel=Ph_sel('all'))
\end{verbatim}

The previous command perfoms a burst search on all photons (\verb|ph_sel=Ph_sel('all')|), with a minimum rate 6 times larger than the background rate (\verb|F=6|) and using 10 consecutive photons to compute the local rate (\verb|m=10|).
A different photon selection, threshold ($F$) or number of photons for rate computation $m$ can be selected by passing a different value. These parameters are generally a good starting point for smFRET analysis but can be tweaked in specific cases.

Note that in the previous burst search no burst size selection is performed (i.e. the minimum bursts size is $m$). 
An additional paramenter $L$ can be passed to apply a threshold on the raw burst size (before any correction). 
We however suggest to perform a more accurate burst size selection as shown in the next section~\ref{sec:burstsel}.

In us-ALEX there are 3 important correction parameters: gamma factor, spectral leakage and 
acceptor direct excitation~\cite{Lee_2005}. These correction can be applied simply setting the respective
Data attributes:

\begin{verbatim}
d.gamma = 0.85
d.leakage = 0.04
d.dir_ex = 0.08
\end{verbatim}

These attributes can be assigned either before or after the burst search. In the latter case, the burst data is
automatically updated using the newly assigned correction values.

Sometimes it may be useful to specify a fixed threshold, instead 
of a threshold derived from the background rate like in the previous example. In this case, instead of $F$ we can use the argument \verb|min_rate| that specifies a threshold in Hertz. For example, a burst search with a 50~kHz
threshold can be perfoemed as follows:

\begin{verbatim}
d.burst_search(rate_min=50e3, m=10, ph_sel=Ph_sel('all'))
\end{verbatim}

Finally, to perform a DCBS burst search (or in general an AND-gate burst search, 
see section~\ref{sec:burstsearch_intro}) the plugin function 
\href{http://fretbursts.readthedocs.org/en/latest/plugins.html#fretbursts.burstlib_ext.burst_search_and_gate}{\verb|burst_search_and_gate|} 
can be used like in the following example:

\begin{verbatim}
d_dcbs = bext.burst_search_and_gate(d, F=6, m=10)
\end{verbatim}

Note that in this case a new \verb|Data| variable is returned (\verb|d_dcbp|) containing all the data and the results of the new burst search. In order to save RAM, FRETBursts shares the timestamps and detectors arrays between different copies of a \verb|Data| object (for example \verb|d| and \verb|d_dcbs|), while all the other data (including background and burst data) is copied. 

The function \verb|burst_search_and_gate| accepts additional arguments \verb|ph_sel1| and \verb|ph_sel2| 
used to specify different photons streams. The default values 
(\verb|ph_sel1 = Ph_sel(Dex='DAem')| and \verb|ph_sel2 = Ph_sel(Aex='Aem')|) correspond to the classical DCBS 
(see section~\ref{sec:burstsearch_intro}).

\subsection{Burst selection}
\label{sec:burstsel}

After burst search it is common to select bursts according to different criteria, among which one of the most common is the burst size.

For example, to select bursts with more than 100 photons (after background correction) detected during the donor excitation periods we can write:

\begin{verbatim}
ds = Sel(d, select_bursts.size, th1=100)
\end{verbatim}

In the previous command a new Data variable (\verb|ds|) containing the selected bursts is created. As mentioned before the new object will share the photon data arrays with the original object (\verb|d|) in order to minimize the RAM consumption.

Looking at the previous command, we notice that the \verb|Sel| function requires a "selection criterium" (a python function) as second argument, while all the remaining arguments are passed to the selection funtion. The module \verb|select_bursts| contains several built-in selection functions like
selecting an area on the E-S ALEX histogram (\verb|select_bursts.ES|), 
selecting bursts based on thei duration (\verb|select_bursts.width|) and so on. New criteria can be implementes by defining a new selection function, usually a couple of lines long (see the \href{https://github.com/tritemio/FRETBursts/blob/master/fretbursts/select_bursts.py}{\verb|select_bursts| module} for examples).

Finally note that different criteria can be combined by applying them
in sequence. For example with the following commands

\begin{verbatim}
ds = Sel(d, select_bursts.size, th1=50, th2=200)
dsw = Sel(ds, select_bursts.with, th1=0.5e-3, th2=3e-3)
\end{verbatim}

the variable \verb|dsw| will contain all the bursts with sizes between 50 and 200 photons, with duration between 0.5 and 3~ms.

\subsubsection{Burst size selection}

In the previous section we used a defition of "burst size" as the total number of detected counts in the donor and in the acceptor channel during donor excitation periods. 

We can modify the selection command in order to also include photons detected in the acceptor channels during acceptor excitation periods. This is achieved passing the boolean flag \verb|add_naa=True| to the selection function as follows:

\begin{verbatim}
ds = Sel(d, select_bursts.size, th1=100, add_naa=True)
\end{verbatim}

Another important parameter in defining the burst size is the gamma-factor, that is the unbalance between the donor and the acceptor channel. The gamma-factor is used to correct for the different quantum yield between D and A fluorophores and the different photon-detection efficiency between the D and A channels. Neglecting the effect of gamma-factor on the burst size leads to a biased burst selection, especially if $\gamma$ significantly differ from 1. 

To include the effect of $\gamma$ on the burst size and obtain a "fair" burst selection (i.e. a selection that does not favor high or low FRET states) we
need to pass the argument \verb|gamma| (or \verb|gamma1|) like in the following example:

\begin{verbatim}
ds = Sel(d, select_bursts.size, th1=100, gamma=0.65)
\end{verbatim}

For more information on burst size selection refer to the \href{http://fretbursts.readthedocs.org/en/latest/burst_selection.html#fretbursts.select_bursts.size}{\verb|select_bursts.size| documentation}.

\subsection{Population fit}

\begin{itemize}
\item Histogram fit: chose a model, constraints, methods, accuracy
\item KDE: find the maximum
\end{itemize}

\subsubsection{Correction coefficients}

\begin{itemize}
\item Fit D-only and A-only population.
\item Fit gamma factor.
\end{itemize}


\subsubsection{Accurate FRET}

Apply corrections to the bursts vs apply corrections after the fit.

