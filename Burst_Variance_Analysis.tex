\section{Implementing burst variance analysis (BVA)}


\label{sec:bva}
In this section we describe how to implement the burst variance analysis (BVA)~\cite{Torella_2011}.
FRETBurts provides well-tested, general-purpose functions for timestamps and burst data 
manipulation and therefore simplifies implementing custom burst analysis algorithms such as BVA.

\subsection{An Introduction to burst variance analysis (BVA)}
Single-molecule FRET histograms show more information than just mean FRET efficiencies. 
While, in general, several peaks indicate the presence of multiple subpopulations, 
a single peak cannot be a priori associated with a single FRET efficiency,
unless a detailed shot-noise analysis is carried out~\cite{Nir_2006,Antonik2006}.

The width of a FRET efficiency population has a typical lower boundary that is caused due to shot noise driven by low number counting statistics. A broader FRET efficiency distribution can be accounted for a mixture of multiple non-interconverting species with different FRET efficiencies, and/or a single species, interconverting at times comparable to the diffusion time. Burst variance analysis (BVA) is an analysis method for single molecule FRET experiments, developed to detect molecular dynamics~\cite{Torella_2011}. It has been successfully implemented to identify heterogeneities in FRET histograms due to dynamic processes of biomolecules in milliseconds and distinguish them from the cases in which broadening occurs due to a static mixture of non-interconverting species~\cite{Torella_2011, Robb_2013}.

a FRET efficiency (sub-)population originating from a single static FRET efficiency has the minimum width and the sub-bursts acceptor counts ($N_a$) can be modeled as a binomial distribution, 
$N_a \sim \operatorname{Binom} \{n, E\}$, where $n$ is the number of photons in each sub-burst and 
$E$ is the FRET efficiency (In practice, Proximity ratio is used as $E$, instead of FRET efficiency). Since, $N_a$ follow a binomial distribution, as modeled, and $E = N_a/n$, we expect the standard deviation of $E$ for the sub-bursts to be distributed according to eq.~\ref{eq:binom_std}. 
This is an approximation because background counts (both from sample and detector's dark counts) add additional variance that is not taken into account. However, the approximation works well in practical cases because the background contribution
is normally a small fraction of the total number of counts (therefore it marginally contributes to the variance).

\begin{equation}
\label{eq:binom_std}
\operatorname{Std(\textit{E})} = {\sqrt{\frac{E(1 - E)}{n}}}
\end{equation}


BVA analysis consists of four steps: 1) slicing bursts into sub-bursts containing \textit{n} consecutive photons, 2) computing FRET efficiencies of each sub-burst, 3) calculating the empirical standard deviation ($s_E$) of sub-burst FRET efficiencies over the whole burst, and 4) comparing $s_E$ to an expected standard deviation based on shot noise limited distribution~\cite{Torella_2011}. 

If the observed broadening originates from different molecules having distinct FRET efficiencies without dynamics, $s_E$ of each burst is only affected by shot noise and will follow the expected standard deviation curve rationalized by a binomial distribution (see equation 4 in~\cite{Torella_2011}). However, if the observed broadness is due to millisecond dynamics of single species of biomolecules, $s_E$ of each burst is supposed to be larger than the expected standard deviation and sit above the expected standard deviation curve as shown in figure .
Since FRETBursts is based on open source python packages, BVA can be easily built and implemented by FRETBursts with combination of other python packages (see notebook).  
