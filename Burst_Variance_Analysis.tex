\section{Implementing Burst Variance Analysis}

\label{sec:bva}
In this section we describe how to implement the burst variance analysis (BVA)~\cite{Torella_2011}.
FRETBurts provides well-tested, general-purpose functions for timestamps and burst data 
manipulation and therefore simplifies implementing custom burst analysis algorithms such as BVA.

\subsection{An Introduction to burst variance analysis (BVA)}
Single-molecule FRET histograms show more information than just mean FRET efficiencies. 
While, in general, several peaks indicate the presence of multiple subpopulations, 
a single peak cannot be a priori associated with a single FRET efficiency,
unless a detailed shot-noise analysis is carried out~\cite{Nir_2006,Antonik2006}.

The width of a FRET distribution has a typical lower boundary set by shot noise, which is caused by
the statistics of discrete photon-detection events. FRET distributions broader than the shot noise limit, 
can be ascribed to a static mixture of species with slightly different FRET efficiencies, 
or to a specie undergoing dynamic transitions (e.g. interconversion between multiple states,
diffusion in a continuum of conformations, binding-unbinding events, etc...).
By simply looking at the FRET histogram,  in cases when there is single peak broader than shot-noise, 
it is not possible to discriminate between the static and dynamic case.
The BVA method has been developed to address this issue of detecting the presence of dynamics 
in FRET distributions~\cite{Torella_2011}, 
and has been successfully applied to identify biomolecular processes with 
dynamics on the millisecond time-scale~\cite{Torella_2011, Robb_2013}.

The basic idea behind BVA is slicing bursts in sub-bursts with a fixed number of photons $n$,
and comparing the empirical variance of acceptor counts across all sub-bursts in a burst 
with the theoretical shot-noise limited variance, dictated by the Binomial distribution.
An empirical variance of sub-bursts larger than the shot-noise limited value indicates
the presence of dynamics. Naturally, since the estimation of the sub-bursts variance is affected
by uncertainty, BVA analysis provides and indication of an higher or lower probability
of observing dynamics.

In a FRET (sub-)population distribution originating from a single static FRET efficiency,
the sub-bursts acceptor counts $N_a$ can be modeled as a Binomial-distributed random variable 
$N_a \sim \operatorname{Binom} \{n, E\}$, where $n$ is the number of photons in each sub-burst and 
$E$ is the estimated population FRET efficiency. Note that, without approximation, we can replace 
E with PR and use the uncorrected counts. This is possible because, regardless of the 
molecular FRET efficiency, the detected counts are partitioned between donor and acceptor channel
according to a Binomila distribution whit a $p$ parameter equal to PR.
The only approximation done here and in the following paragraphs is neglecting the presence background
(a reasonable approximation since the backgrounds counts are in general a very small fraction of
the total counts). 
We refer the interested reader to~\cite{Torella_2011} for further discussion.

If $N_a$ follows a binomial distribution, the random variable $E = N_a/n$,
has standard deviation reported in eq.~\ref{eq:binom_std}. 

\begin{equation}
\label{eq:binom_std}
\operatorname{Std(\textit{E})} = {\sqrt{\frac{E(1 - E)}{n}}}
\end{equation}

\subsection{BVA implementation}
BVA analysis consists of four steps: 1) slicing bursts into sub-bursts containing a constant number of consecutive photons,~\textit{n}, 2) computing FRET efficiencies of each sub-burst, 3) calculating the empirical standard deviation ($s_E$) of sub-burst FRET efficiencies over the whole burst, and 4) comparing $s_E$ to the expected standard deviation of a shot-noise limited distribution~(eq.~\ref{eq:binom_std}). 

If, as in figure~\ref{fig:bva_static}, the observed distribution originates a static mixture 
of FRET efficiencies (without dynamics), 
$s_E$ of each burst is only affected by shot noise and will follow the expected standard deviation curve based on eq.~\ref{eq:binom_std}. Conversely, if the observed distribution is due to dynamics of single species of biomolecules as in figure~\ref{fig:bva_dynamic}, $s_E$ of each burst will be larger than the expected standard deviation and will sit above the expected standard deviation curve as shown in figure~\ref{fig:bva_dynamic}.
