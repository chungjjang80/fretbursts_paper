\section{Implementing Burst Variance Analysis}

FRET histograms could show more information than just mean FRET efficiencies. Broad FRET distributions might be attributed to the mixture of multiple species with static but different FRET efficiencies, single species with dynamic fluctuations between multiple FRET states, or a combination of the two cases. Burst Variance Analysis (BVA) is an analysis method for single molecule FRET experiments, developed to detect molecular dynamics~\cite{Torella_2011}. It has been successfully implemented to identify heterogeneities in FRET histograms due to dynamic processes of biomolecules in millisecond time scale~\cite{Torella_2011, Robb_2013}.

BVA analysis consists of three steps: 1) slicing a bursts by \textit{n} consecutive photons into sub-bursts, 2) computing FRET efficiencies of each sub-burst and calculating an observed standard deviation ($s_E$) of these FRET efficiencies from a mean FRET efficiency for the whole burst, and 3) comparing $s_E$ to an expected standard deviation based on shot noise limited distribution~\cite{Torella_2011}. 

If the observed broadness originates from different molecules having distinct FRET efficiencies without dynamics, $s_E$ of each burst is only affected by shot noise and will follow the expected standard deviation curve rationalized by a binomial distribution (see equation 4 in~\cite{Torella_2011}). However, if the observed broadness is due to millisecond dynamics of single species of biomolecules, $s_E$ of each burst is supposed to be larger than the expected standard deviation and sit above the expected standard deviation curve as shown in figure .
Since FRETBursts is based on open source python packages, BVA can be easily built and implemented by FRETBursts with combination of other python packages (see notebook).  



see section~\ref{sec:ph_streams}