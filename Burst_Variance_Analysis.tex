\section{Implementing Burst Variance Analysis}

\label{sec:bva}
In this section we describe how to implement the burst variance analysis (BVA)~\cite{Torella_2011}.
FRETBurts provides well-tested, general-purpose functions for timestamps and burst data 
manipulation and therefore simplifies implementing custom burst analysis algorithms such as BVA.

\subsection{An Introduction to burst variance analysis (BVA)}
Single-molecule FRET histograms show more information than just mean FRET efficiencies. 
While, in general, several peaks indicate the presence of multiple subpopulations, 
a single peak cannot be a priori associated with a single FRET efficiency,
unless a detailed shot-noise analysis is carried out~\cite{Nir_2006,Antonik2006}.

The width of a FRET distribution has a typical lower boundary set by shot noise, which is caused by
the statistics of discrete photon-detection events. FRET distributions broader than the shot noise limit, 
can be ascribed to a static mixture of species with slightly different FRET efficiencies, 
or to a specie undergoing dynamic transitions (e.g. interconversion between multiple states,
diffusion in a continuum of conformations, binding-unbinding events, etc...).
By simply looking at the FRET histogram,  in cases when there is single peak broader than shot-noise, 
it is not possible to discriminate between the static and dynamic case.
The BVA method has been developed to address this issue of detecting the presence of dynamics 
in FRET distributions~\cite{Torella_2011}, 
and has been successfully applied to identify biomolecular processes with 
dynamics on the millisecond time-scale~\cite{Torella_2011, Robb_2013}.

The basic idea behind BVA is slicing bursts in sub-bursts with a fixed number of photons $n$,
and comparing the empirical variance of acceptor counts across all sub-bursts in a burst 
with the theoretical shot-noise limited variance, dictated by the Binomial distribution.
An empirical variance of sub-bursts larger than the shot-noise limited value indicates
the presence of dynamics. Naturally, since the estimation of the sub-bursts variance is affected
by uncertainty, BVA analysis provides and indication of an higher or lower probability
of observing dynamics.

In a FRET (sub-)population distribution originating from a single static FRET efficiency,
the sub-bursts acceptor counts $N_a$ can be modeled as a Binomial-distributed random variable 
$N_a \sim \operatorname{Binom} \{n, E\}$, where $n$ is the number of photons in each sub-burst and 
$E$ is the estimated population FRET efficiency. Note that, without approximation, we can replace 
E with PR and use the uncorrected counts. This is possible because, regardless of the 
molecula FRET efficiency, the detected counts are partitioned between donor and acceptor channel
according to a Binomila distribution whit a $p$ parameter equal to PR.
The only approximation done here and in the following paragraphs is neglecting the presence background. 
We refer the interested reader to~\cite{Torella_2011} for further discussion.

the PR instead of the corrected The same considerations holds if, instead of $E$ 
we use the PR grated that we used the uncorrected acceptor counts in this case.  and Note that, we can also substitute $E$ with $PR$ Proximity ratio is used as $E$, instead of FRET efficiency). Since, $N_a$ follow a binomial distribution, as modeled, and $E = N_a/n$, we expect the standard deviation of $E$ for the sub-bursts to be distributed according to eq.~\ref{eq:binom_std} ~\cite{Torella_2011}. 
This is an approximation because background counts (both from sample and detector's dark counts) add additional variance that is not taken into account. However, the approximation works well in practical cases because the background contribution
is normally a small fraction of the total number of counts (therefore it marginally contributes to the variance).

\begin{equation}
\label{eq:binom_std}
\operatorname{Std(\textit{E})} = {\sqrt{\frac{E(1 - E)}{n}}}
\end{equation}

BVA analysis consists of four steps: 1) slicing bursts into sub-bursts containing a constant number of consecutive photons,~\textit{n}, 2) computing FRET efficiencies of each sub-burst, 3) calculating the empirical standard deviation ($s_E$) of sub-burst FRET efficiencies over the whole burst, and 4) comparing $s_E$ to an expected standard deviation based on shot noise limited distribution~\cite{Torella_2011}. 

If the observed broadening originates from different molecules having distinct FRET efficiencies without dynamics, $s_E$ of each burst is only affected by shot noise and will follow the expected standard deviation curve based on eq.~\ref{eq:binom_std} (Fig.~\ref{fig:bva_static}). However, if the observed broadness is due to millisecond dynamics of single species of biomolecules, $s_E$ of each burst is supposed to be larger than the expected standard deviation and sit above the expected standard deviation curve as shown in figure~\ref{fig:bva_dynamic}.
