\subsection{Burst selection}
\label{sec:burstsel}

After burst search it is common to select bursts according to different
criteria, among which one of the most common is the burst size.

For example, to select bursts with more than 100 photons (after background
correction) detected during the donor excitation periods one can write:

\begin{verbatim}
ds = Sel(d, select_bursts.size, th1=100)
\end{verbatim}

In the previous command a new \textit{Data} variable (\verb|ds|) containing the selected
bursts is created. As mentioned before the new object will share the photon data
arrays with the original object (\verb|d|) in order to minimize the RAM
consumption.

Looking at the previous command, we notice that the
\href{http://fretbursts.readthedocs.org/en/latest/burst\_selection.html#fretbursts.burstlib.Sel}{\texttt{Sel} function}
requires a "selection criterium" (a python function) as second
argument; all the remaining arguments are passed to the selection function. The
module \verb|select_bursts| contains numerous built-in selection functions, for
example to
select a region on the E-S ALEX histogram (\verb|select_bursts.ES|), 
to select bursts based on their duration (\verb|select_bursts.width|) and so on.
New criteria can be easily 
implemented by defining a new selection function, usually not longer than a
couple of lines (see the
\href{https://github.com/tritemio/FRETBursts/blob/master/fretbursts/select\_bursts.py}{\texttt{select\_bursts} module} for several examples).

Finally note that different criteria can be combined by applying them
in sequence. For example with the following commands

\begin{verbatim}
ds = Sel(d, select_bursts.size, th1=50, th2=200)
dsw = Sel(ds, select_bursts.width, th1=0.5e-3, th2=3e-3)
\end{verbatim}

the variable \verb|dsw| will contain all the bursts with sizes between 50 and
200 photons, with duration between 0.5 and 3~ms.

\subsubsection{Burst size selection}

In the previous section we used a definition of "burst size" as the total number
of detected counts in the donor and in the acceptor channel during donor
excitation periods. 

We can modify the selection command in order to also include photons detected in
the acceptor channels during acceptor excitation periods. This is achieved
passing the boolean flag \verb|add_naa=True| to the selection function as
follows:

\begin{verbatim}
ds = Sel(d, select_bursts.size, th1=100, add_naa=True)
\end{verbatim}

Another important parameter in defining the burst size is the gamma-factor, i.e.
the unbalance between the donor and the acceptor channels. The gamma-factor is
used to correct for the different quantum yield between D and A fluorophores and
the different photon-detection efficiency between the D and A channels.
Neglecting the effect of gamma-factor on the burst size leads to a biased burst
selection, especially if $\gamma$ significantly differ from 1. 

To include the effect of $\gamma$ on the burst size and obtain a "fair" burst
selection (i.e. a selection that does not favor high or low FRET states) we
need to pass the argument \verb|gamma| (or \verb|gamma1|) like in the following
example:

\begin{verbatim}
ds = Sel(d, select_bursts.size, th1=100, gamma=0.65)
\end{verbatim}

For more information on burst size selection refer to the
\href{http://fretbursts.readthedocs.org/en/latest/burst_selection.html#fretbursts.select\_bursts.size}{\texttt{select\_bursts.size} documentation}.


