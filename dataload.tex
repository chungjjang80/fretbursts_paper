\section{smFRET burst analysis}
\label{sec:analysis}

\subsection{Loading the data}
\label{sec:dataload}

Currently, FRETBursts supports loading data from a few file formats: SM files
(a binary format saved by a common LabVIEW program used in smFRET setups), 
SPC (a binary format used by TCSPC Becker \& Hickl cards) and 
\href{http://photon-hdf5.readthedocs.org/}{Photon-HDF5}  an open binary 
format single-molecule data based on HDF5. Support for additional file formats
such as the PicoQuant pt3 files can be easily added on user request.

We encourage adopting \href{http://photon-hdf5.readthedocs.org/}{Photon-HDF5}
as a fast and space efficient format that assure long-term accessibility 
to the data and ease data sharing between different programs. All the 
FRETBursts example data files are in Photon-HDF5 and can be opened with 
standard viewers such as 
\href{http://www.hdfgroup.org/products/java/hdfview/}{HDFView}. An additional data file 
can be found here[multi-spot paper data files].

The \href{http://fretbursts.readthedocs.org/en/latest/loader.html}{\texttt{loader} module}
contains all the functions to load the different file formats. 
In particular, to load a Photon-HDF5 file use:

\begin{verbatim}
d = loader.hdf5(file_name)
\end{verbatim}

where \verb|file_name| is a string containing the file path. The function
\verb|load.hdf5| returns an object \verb|d| of type 
\href{http://fretbursts.readthedocs.org/en/latest/data_class.html}{\texttt{Data}}) 
containing all the measurement info (see section~\ref{sec:data_intro}).

Similar functions in the 
\href{http://fretbursts.readthedocs.org/en/latest/loader.html}{\texttt{loader} module} 
allows to load the other supported file formats.

\subsubsection{Load data manually}

In case the data is available in a format not directly supported by 
FRETBursts it is possible to manually create a \verb|Data| variable. 
For example, for non-ALEX smFRET data, two arrays of same length are 
needed: the timestamps and the acceptor-mask. The timestamps need to be 
an int64 numpy array containing the recorded photon timestamps in arbitrary 
units (usually dictated by the acquisition hardware clock period). 
The acceptor-mask needs to be a numpy boolean array that is \verb|True| 
when the corresponding timestamps comes from the acceptor channel and 
\verb|False| when it comes from the donor channel. Having these arrays a 
\verb|Data| object can be manually created with:

\begin{verbatim}
d = Data(ph_times_m=[timestamps], A_em=[acceptor_mask], 
         clk_p=10e-9, ALEX=False, nch=1, fname='file_name')
\end{verbatim}

In the previous example, we set the timestamp unit (\verb|clk_p|) to 10~ns 
and we specify that the data is not from an ALEX measurements. Creating 
\verb|Data| objects for ALEX and nsALEX measurements follows the same lines. 
We point the interested readers to the 
\href{https://github.com/tritemio/FRETBursts/blob/master/fretbursts/loader.py}{loader module source code} 
for other examples. 

\subsection{Alternation parameters}
\label{sec:alternation}

For us-ALEX and ns-ALEX data it is necessary to identify the different 
alternation periods for donor and acceptor excitation.

The functions 
\verb|plot_alternation_hist| (\href{http://fretbursts.readthedocs.org/en/latest/plots.html#fretbursts.burst\_plot.plot\_alternation\_hist}{documentation})
and \verb|plot_alternation_hist_nsalex| (\href{http://fretbursts.readthedocs.org/en/latest/plots.html#fretbursts.burst\_plot.plot\_alternation\_hist\_nsalex}{documentation})
plot the alternation histogram (or the nanotime histogram)
with the currently selected donor and acceptor excitation periods
(if no selection has been made yet, default values will be shown).
Figure~\ref{fig:altern_hist} shows a typical alternation histogram for
a µs-ALEX measurement the highlighted donor and acceptor excitation periods.

To change the period definition shown in figure~\ref{fig:altern_hist}
the user can execute:

\begin{verbatim}
    d.add(D_ON=(2850, 580), A_ON=(900, 2580))
\end{verbatim}

where \verb|D_ON| and \verb|A_ON| are tuples representing 
the start-stop range for the donor and acceptor excitation periods 
in timestamps units. After changing the parameters a new plot
alternation plot will show the new updates period selections.

When the alternetion period selection is satisfactory it can
be applied with:

\begin{verbatim}
    loader.usalex_apply_period(d)
\end{verbatim}

After this command, only the photons inside excitation periods
will be present in \verb|d| and to apply a different alternation 
period definition the file must be reloaded.

