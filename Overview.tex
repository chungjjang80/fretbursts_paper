\section{FRETBursts Overview}
\label{sec:overview}

\subsection{Features}


FRETBursts can analyze smFRET measurements
from one or multiple excitation spots~\cite{Ingargiola_2013}. The supported
excitation schemes include single laser, alternating laser excitation (ALEX)
with either CW lasers (µs-ALEX \cite{Kapanidis_2005})
or pulsed lasers (ns-ALEX~\cite{Laurence_2005} or
pulsed-interleaved excitation (PIE)~\cite{M_ller_2005}).

The software implements both standard and novel algorithms for smFRET data analysis
including background estimation as a function of time (including background accuracy
metrics), sliding-window burst search~\cite{Eggeling_1998}, 
dual-channel bursts search~\cite{Nir_2006} and
modular burst selection methods based on user-defined criteria
(including a large set of pre-defined selection rules). Novel features include burst size
selection with gamma-corrected burst size, burst weighting, burst search with background-
dependent threshold (in order to guarantee a minimal single-to background 
ratio~\cite{Michalet_2012}).
Moreover, a large set of FRET distribution fitting options are provided including
FRET distribution fitting through (1) histogram fitting (with arbitrary model functions),
(2)  non-parametric weighted \textit{Kernel Density Estimation} (KDE), (3) weighted
Expectation Maximization (EM), (4) Maximum Likelihood fitting using Gaussian models
or Poisson statistic. Finally FRETBursts includes a large number of
predefined and customizable plot functions which (thanks to the \textit{matplotlib}
graphic library) produce publication quality plots in a wide range of formats.


\subsection{FRETBursts and Reproducibility}
FRETBursts has been developed with the goal of facilitating computational reproducibility
of the performed data analysis~\cite{Buckheit_1995}. For this reason,
the preferential way of using FRETBursts is by executing one of the tutorials
which are in the form of \href{http://ipython.org/notebook.html}{Jupyter notebooks}.
Jupyter (formerly IPython) notebooks are web-based documents which contain both
code and rich text (including equations, hyperlinks, figures, etc...).
FRETBursts tutorials are notebooks which can be re-executed,
modified or used to process new data files with minimal modifications.
The ``notebook workflow''~\cite{Shen_2014} not only facilitates
the description of the analysis (by integrating the code in a rich document)
but also greatly enhance its reproducibility by storing an execution trail
that includes software versions, input files, parameters, commands and all
the analysis results (text, figures, tables, etc...).

The Jupyter Notebook environment streamlines FRETBursts execution (compared to
a traditional script and terminal based approach) and therefore
FRETBursts can be used even without prior python knowledge.
The user only needs to get familiar with the
notebook graphical environment, in order to be able to navigate and run the notebooks.
The list of FRETBursts notebooks can be found in the
\verb|FRETBursts_notebooks| repository on GitHub
(\href{https://github.com/tritemio/FRETBursts\_notebooks}{link}).


\subsection{Development Style}

FRETBursts (and the entire python ecosystem it depends on) is open source
and the source code is fully available for any scientist to study,
review and modify.
The authors encourage users to use GitHub issues for questions, discussions
and bug reports, and to submit patches through GitHub pull requests.

In order to minimize the likelihood of bugs and erroneous results, FRETBursts is developed
following modern software engineering techniques such
as defensive programming, unit testing, regression testing and continuous integration~\cite{Wilson_2014}.

The open source nature of FRETBursts and of the python ecosystem,
not only makes it a more transparent, reviewable platform
for scientific data analysis, but also allows
to leverage state-of-the-art online services as \href{http://https://github.com}{GitHub} for hosting,
issues tracking and code reviews,
\href{https://travis-ci.org}{TravisCI} for continuous integration
(i.e. automated test suite execution on multiple platforms after each commit)
and \href{https://readthedocs.org/}{ReadTheDocs.org} for automatic documentation building and hosting.
All these services would be extremely costly, if available \textit{tout court},
for a proprietary software or platform~\cite{Freeman_2015}.
number of photons detected by the acceptor channel
during acceptor excitation period (present only in ALEX measurementscolumnnormally neglected as
